%
\begin{acknowledgments}
% 
This work was supported by the United States Geological Survey (USGS) through the award ``Evaluation of the southern California seismic velocity models through ground motion simulation and validation of past earthquakes'' (G14AP00034); and by the Southern California Earthquake Center (SCEC) through the award ``Evaluation of CVM-SI.26 perturbations integration involving undergraduate computer science and graduate earth sciences and engineering students'' (14032). Additional support was provided through the U.S.~National Science Foundation (NSF) award ``SI2-SSI: A Sustainable Community Software Framework for Petascale Earthquake Modeling'' (ACI-1148493). This research is also part of the Blue Waters sustained-petascale computing project, which is supported by NSF (awards OCI-0725070 and ACI-1238993) and the state of Illinois. Blue Waters is a joint effort of the University of Illinois at Urbana-Champaign and its National Center for Supercomputing Applications (NCSA). Computational support was possible through PRAC allocations supported by NSF awards ``Petascale Research in Earthquake System Science on Blue Waters (PressOnBlueWaters)'' (ACI-0832698); and ``Extending the spatiotemporal scales of physics-based seismic hazard analysis'' (ACI-1440085). SCEC is funded by NSF Cooperative Agreement EAR-1033462 and USGS Cooperative Agreement G12AC20038. The SCEC contribution number for this paper is \textcolor{red}{PEDNING}.
% 
\end{acknowledgments}
