%SRL Article

\documentclass{gji}
\usepackage{timet}

\title[SRL\ J.\ Int.]
  {\par1. A comparative study between CVM-SI.26 and CVMS-4 Velocity Model \par 2.The Comparison of Two Versions of Southern California Velocity Models: CVM-SI.26 and CVMS-4\par 3. Evaluation of latest Southern California Velocity Models, CVM-SI.26, through Simulation and Validation of Multiple Historical Events\par 4. A comparative study between Two Versions of Southern California Velocity Models: CVMS-4 and the latest one, CVM-SI.26, through Simulation and Validation of Multiple Historical Events }
\author[-]
  {B.L.N. Kennett$^1$\thanks{Pacific Region Office, GJI} \\
  $^1$ Research School of Earth Sciences, Australian National
    University, Canberra ACT \emph{0200}, Australia
  }
\date{Received 1998 December 18; in original form 1998 November 22}
\pagerange{\pageref{firstpage}--\pageref{lastpage}}
\volume{200}
\pubyear{1998}

%\def\LaTeX{L\kern-.36em\raise.3ex\hbox{{\small A}}\kern-.15em
%    T\kern-.1667em\lower.7ex\hbox{E}\kern-.125emX}
%\def\LATeX{L\kern-.36em\raise.3ex\hbox{{\Large A}}\kern-.15em
%    T\kern-.1667em\lower.7ex\hbox{E}\kern-.125emX}
% Authors with AMS fonts and mssymb.tex can comment out the following
% line to get the correct symbol for Geophysical Journal International.
\let\leqslant=\leq

\newtheorem{theorem}{Theorem}[section]

\begin{document}

\label{firstpage}

\maketitle

\section{Introduction}

Among the several alternatives to earthquake simulations, physics-based, deterministic approach to seismic hazard analysis (SHA) 
have drawn significant attention from seismologists and earthquake engineers. These simulations use numerical approaches such as the finite-element (FE), finite-difference (FD), and high-order FE or spectral element methods (e.g., Frankel and Vidale, 1992; Olsen et al., 1995; Graves, 1996; Bao et al., 1998; Seriani, 1998; K�ser and Dumbser, 2006). 
With the growth of high-performance computing (HPC) facilities and applications, the implementation of those numerical methods for large scale
three-dimensional (3D) earthquake simulations and rupture dynamics has progressed considerably  and researchers are now modeling earthquake-related processes at a level of fidelity that was thought not possible just a few years ago (e.g., Komatitsch et al., 2004; Aagaard et al., 2008; Graves, 2008; Olsen et al., 2009; Bielak et al., 2010; Cui et al., 2010). This has allowed us to deliver products of relevant application in seismology and engineering, such as families of synthetics for scenario earthquakes (e.g., Jones et al., 2008) and physics-based seismic hazard maps (e.g., Graves et al., 2011). The complexity and multi-scale nature of earthquakes, however, still poses immense challenges like the accuracy of the models which is heavily depend on the source and velocity models used as input. \par

Significant advances have been made in 3D waveform tomography (e.g., Tape et al.,
2009; Chen et al., 2007a, 2010; Lee et al., 2011; Lee and Chen 2013). In particular, Chen et al. (2007b) and Lee et
al., (2013) have used local small and medium-sized earthquakes and ambient noise Green's functions in full 3D
waveform tomographic inversions for the crustal structure of the Los Angeles and greater Southern California regions. Recently obtained results from a full three-dimensional (3D) tomography inversion for the crustal
structure of the Southern California region provide perturbation datasets that are being integrated back
into the original velocity model used as reference, SCEC Community
Velocity Model (CVM-S), to build a new model called CVM-SI.26. These perturbations
are, however, available only at a coarse resolution (500 m) and the scheme put forward to integrate them
with CVM-S needs to be tested before the new model CVM-SI.26 can be used for querying information at arbitrary
points.  \par

The main goal here is to provide a better understanding of the improvements in the new model
CVM-SI.26 when compared to the original model CVM-S4. There have been some works in this area with considering one or two earthquake ( Taborda 2013; En ju Lee 2014). This article evaluates the overall improvements obtained with the inversion and tests the correctness of the integration schemes when used at higher resolution than that of the inversion. This evaluation is done systematically through the forward simulation and validation of multiple small- and moderate-sized events using forward simulations in order to avoid biases from individual simulations. The simulations are done for a maximum frequency of 1 Hz and a minimum shear wave
velocity of 200 m/s using Hercules, an octree-based finite-element parallel earthquake ground motion simulator.
The earthquake ruptures are modeled assuming a kinematic point-source (double-couple) representation.
Synthetics on about 50 stations per event are compared with data recorded during the earthquakes
and downloaded from the Southern California Earthquake Data Center. The seismograms are processed
and compared with the synthetics in the range of 0.1-1.0 Hz using a modified version of the Anderson Goodness-
of-Fit (GOF) method. This method combines various comparison criteria that are meaningful to both engineers
and seismologists, and has been satisfactorily used in previous validation efforts. We compare the regional
distribution of the GOF results for all the earthquakes and all the models and draw conclusions on the
consistency observed in the results. Here the results show overall higher goodness of fit scores for CVM-SI.26 model.  



\section{Material Models: CVM-S4 and CVM-SI.26 }

The ability to simulate seismograms depends on having accurate 3D material model. Material models, also called seismic velocity or crustal models, are a key element in the simulation process because, while being used to construct the mesh and determine the properties of mesh elements that constitute the numerical representation of the system,
their properties determine in good part the outcome of the simulation. 

Several material models have been developed and improved over the years for the regions of southern and northern California (e.g. Graves 1996b; Magistrale et al. 1996; Hauksson & Haase 1997; Magistrale et al. 2000; Kohler et al. 2003; Suss & Shaw 2003; Brocher 2005). Our focus here is on subsets of SCEC's Community Velocity Model (CVM), version 4.0, and version 4.26.  \par


\subsection{CVM}

The CVM is distributed as a Fortran code that reads points specified
by latitude, longitude and depth, and writes out V_{p}, V_{s}, and
density values at those points. Since the Fortran code has to
combine data from many different associated files, its performance
for the construction of large meshes needed for FDM and FEM
models is poor. To overcome this problem and to unify to the maximum
possible extent the data used for the grids and mesh generation
processes of each modelling group, we stored a discretized version
of the CVM (for the region of interest) in an octant tree database
form, called "etree" (Tu et al. 2003). The etree library is known
for its efficiency in query retrieval time and has been used successfully
in past earthquake simulations (e.g. Akcelik et al. 2003;
Rodgers et al. 2008). The discretization process, described in detail
in Taborda et al. (2007), was done in a way that guarantees that
the discretized version, hereafter called CVM-Etree, is appropriate
for simulations up to a maximum frequency f_{max} = 1 Hz and a
minimum shear wave velocity V_{smin}
= 200 m/s, with at least eight
points per wavelength (p = 8). 

The resolution of the CVM-Etree
decreases with depth while maintaining the rule eo ? (Vs/ f max)/p,
where eo is the octant edge size. This results in a minimum octant
at the surface with eo = 18.3 m?where there is maximum
variability?and a maximum octant with eo = 9.2 km at 80 km in
depth?where the crustal structure is fairly homogeneous. Fig. 5
shows the southern California basins and other geological features
as represented in the model, and the basins floor and shear wave
velocity structure of the resulting CVM-Etree at various depths. \par


\subsection{CVM-S}

Crustal models that specify the lateral and depth variations of seismic velocities have been developed for southern California upon the initial model assembled by Magistrale et al. (1996), and later extended and improved by Magistrale et al. (2000) and Kohler et al. (2003). In Magistrale et al. (2000), the P-wave velocity (V_{p}) of the major southern California basins (Los Angeles basin,Ventura basin, San GabrielValley, San Fernando Valley, Chino basin, San Bernardino Valley and the Salton Trough) was determined primarily by application of empirical rules based on depths and ages estimated for four geological horizons and calibrated for southern California with data from deep boreholes. The density and Poisson's ratio (and S-wave velocity, V_{s}) were estimated from V_{p} using empirical relationships. 

Outside and below the basins, CVM-S uses a 3D seismic
tomography model (Hauksson, 2000) and an upper-mantle
model based on on the inversion of teleseismic travel time
residuals obtained from three temporary passive experiments and stations of the Southern California Seismic Network.(Kohler et al., 2003).
Inside the basins, it uses data from shallow and deep boreholes,
oil wells, gravity observations, seismic refraction surveys
surveys,
and other empirical rules based on the depths and ages
estimated for a set of geological horizons.


When queried at a particular longitude, latitude, and
depth, CVM-S provides the values of VP, VS, and ? for the
specific query point. In general, CVM-S operates by depth,
in reference to the free surface, but does not provide information
about the topography. We ignore the free-surface
elevation and model the region in the simulation domain
by flattening the topography (see squashed topography in
Aagaard et al., 2008). We access the information stored in
CVM-S through the Unified Community Velocity Model
(UCVM) software framework developed by SCEC (Small
et al., 2011; Gill et al., 2013) and use additional utilities provided
by UCVM to facilitate the simulation process (see Simulation
Approach and Models section).\par





\subsection{CVM-S4}

The latest release by SCEC, CVM version 4.0, has a new San Bernardino Valley basement, based on recent inversions of gravity data confirmed by comparison to a seismic reflection line. The depth to basement of the Salton Trough was extended to the northern Mexican territory and redefined by a combination of seismic refraction surveys, inversion
of gravity observations, surface geology, and boreholes. CVM version 4.0 also has a new V p-density relationship based on density
measurements from oil well samples in the Los Angeles basin and the San Gabriel Valley, geotechnical boreholes throughout southern
California, and 12 oil wells along the Los Angeles Region Seismic Experiment lines. \par







\cite{r3} compares sections through two of the SCEC models?CVM-S4 and CVM-S4.26 with 2D tomographic models
derived from a dense collection of reflection/refraction data along the two lines of the Los Angeles region seismic experiments
(LARSE), LARSE I (Lutter et al., 1999) and LARSE II (Lutter et al., 2004). The models show significant differences at
the basin scales, as well as deeper in the crust.

\begin{figure} [!h]
\centering
\includegraphics[scale=1]{fig1.jpg}
\caption{$Comparisons of P-wave velocity models along the LARSE-I (left) and LARSE-II (right) profiles. From top to bottom, we show
the 2D tomography models obtained using controlled-source travel-time data and the cross-section views through CVM-S4.26 and CVM-S4. The vertical axes on the velocity model plots have been exaggerated by a factor of 2.0 for the LARSE-I profile and a factor
of 2.5 for the LARSE-II profile. The geographic maps show the locations of the LARSE-I (left) and LARSE-II (right) profiles and the epicenters
of the Encino (yellow star) and the La Habra (red star) earthquakes. WF, Whittier fault; SMFZ, Sierra Madre fault zone; SAF, San
Andreas fault; SSF, Santa Susana fault; SGF, San Gabriel fault; GF, Garlock fault.$ \cite{r1}}
\label{Figure 1}
\end{figure}


In CVM-S, the seismic velocities within major basins were determined
mainly from the age and depths of the sediments using
empirical relations. The latest official release is version 4
(CVM-S4), which includes a geotechnical layer constrained by
sonic log data (Magistrale et al., 2000), a variable-depth Moho
determined from receiver functions (Zhu and Kanamori, 2000),
and an upper-mantle velocity model from Moho to about
100 km depth (Kohler et al., 2003). \par

CVM-S4.26 is the 26th iterate of a full 3D tomographic
(F3DT) inversion procedure (Lee et al., 2014). The procedure
started with CVM-S4 and successively improved the fit to datasets
that eventually included about 550,000 differential waveform
measurements at frequencies up to 0.2 Hz, obtained from
about 38,000 earthquake seismograms and 12,000 ambientnoise
Green?s functions. Navigation through this nonlinear
iterative process involved two types of F3DT inversion methods:
the AW-F3DT, which backpropagates the misfits between
observed and synthetic seismograms from the receivers to
image structures (Tarantola, 1984; 1988; Pratt, 1990; Tromp
et al., 2005), and the scattering-integral method (SI-F3DT),
which calculates and stores the sensitivity kernels of each misfit
measurement and solves the Gauss?Newton normal equation
using the least-squares algorithm (Zhao et al., 2005, Zhao et al.,
2006; Chen et al., 2007).
In each inversion step, synthetic seismograms for the updated
model were calculated using the Olsen (1994) fourthorder
staggered-grid finite-difference code, which has been
optimized for massively parallel computations (Cui et al.,
2010), and differential waveform measurements were made between
the observed seismograms and these synthetics, accounting
for the nonlinearity of the inversion. High structural
resolution was obtained by included frequency-dependent,
phase-coherent measurements of various seismic phases on all
three components of ground motion. Lee et al. (2014) describe
CVM-S4.26 and demonstrate its excellent fit to observed seismograms
from a large number of well-recorded earthquakes.
The seismograms from the two Los Angeles events analyzed
here were not used in the F3DT inversion. \par




There are currently three alternative CVM-SI.26 models which vary depending on how the perturbations were applied
to the original model. The different versions and the overall process are described in greater detail in the GTL
entry of the SCEC Wiki Web site (http://scec.usc.edu/scecpedia/GTL). The entire process goes beyond the scope of
the proposed research because the ultimate goal is, not only to incorporate the inversion results, but also to extend
the implementation to include geotechnical layers (GTL) used in
other community models (e.g., CVM-H). I refer here only to the
models that reconcile the inversion perturbations?as a first step
in that process, and use the same nomenclature employed in the
Web site. These models are: \par

(2.2.1) This model applies an integration scheme in which
negative perturbations are only used outside the basins, and
positive perturbations are used everywhere.\par

(2.2.2) This scheme applies negative perturbations only if
outside the basins, and disregards positive perturbations
when inside the basins.\par

(2.2.3) This scheme applies both negative and positive perturbations
everywhere, but sets the original value inside the
basins as a floor limit.\par

Here, the term basin refers to any structure with Vs \le1000
m/s adopting the cutoff value used by Lee et al. (2013) to build
the starting model. In all cases, the schemes reverse the process
used to build the starting model first, in order to recapture the
original structures in CVM-S with values of Vs \le 1000 m/s, and
apply the perturbations later (as described). These schemes have
been implemented in the SCEC Unified Community Velocity
Model (UCVM) software framework (Gill et al. 2013), which has
already been used to produce unstructured (etree) meshes (Taborda
et al., 2007; Tu et al., 2003) at resolutions finer than that used in the inversion. Figure 1 shows the Vs profile
at a depth of 50 m in an area that includes all the major basins in the Los Angeles region and its corresponding perturbations1.\par

\begin{figure} [!h]
\centering
\includegraphics[scale=1]{fig2.jpg}
\caption{$Shear wave velocity (in m/s) o the CVM-SI26(2.2.3) model at a depth of 50 m (top) and perturbations with respect to the original CVM-S model (bottom).$ \cite{r2}}
\label{Figure 2}
\end{figure}


\section{Methodology and Parameters:  }

In order to evaluate CVM-SI.26, we need to evaluate the integration scheme and to quantify the improvement (or lack thereof) to the prediction
of the ground motion that results from using the new integrated model. The applied method is to do this
by performing a series of simulations using CVM-SI.26 in contrast to CVM-S for a collection of events. The 29 selected events (3.5 \le M \le 5.5) are a subset of the 160 earthquakes used in the inversion process. The synthetic ground-motion results
from these simulations are compared with records available for these events through the Southern California
Earthquake Data Center (www.data.scec.org). The comparisons are done quantitatively using intensity goodness-of-fit measures  of a modified version of the Anderson Goodness-of-Fit (GOF) method and by direct waveform comparisons.This method combines various comparison criteria that are meaningful to both engineers and seismologists, and has been satisfactorily used in previous validation efforts.  Simulations are done at a maximum resolution frequency higher
than that of the inversion?in order to assess the arbitrariness of the new model?but are kept low enough (0.1 \le
fmax \le 1 Hz) so that it facilitates the execution of as many simulations as possible. \par

\cite{r3}. summarizes the simulation parameters. The etree models were built on Kraken at the National Institute
for Computational Sciences, and the ground-motion simulations were done on Blue Waters at the National Center for
Supercomputing Applications. \cite{r4}.shows the mesh characteristics and simulation performance for each model and simulation.


\begin{figure} [!h]
\centering
\includegraphics[scale=1]{table3.jpg}
\caption{$Selected Events Chractristics$ \cite{r6}}
\label{Table3}
\end{figure}


\begin{figure} [!h]
\centering
\includegraphics[scale=1]{table1.jpg}
\caption{$Simulation Parameters$ \cite{r3}}
\label{Table1}
\end{figure}


\begin{figure} [!h]
\centering
\includegraphics[scale=1]{table2.jpg}
\caption{$Finite-Element Mesh Characteristics and Simulation Performance$ \cite{r4}}
\label{Table2}
\end{figure}


The process includes gathering of data, setup of simulation models, and execution of an
initial simulations which is done at the High Performance Computing Center ofBlue Water which has a cluster large enough to conduct initial simulations.  The approach will consists of simulations of
multiple small- and medium-size events and quantitative validation of synthetics to assess model improvements such as those used in Taborda
and Bielak (2013, 2014).
This requires the use of parallel computing.  The following steps are taken for the process:\par

Step1. Selection of events and data pre-processing: A
total of 160 events were used in the 3D tomographic inversion
done by Lee et al. (2013) (see Figure 4.a). We select a region
similar to that , that is suitable for
evaluating the effects, and select a subset of these events
(~30). Table 1 presents a summary of properties of each of the selected earthquake. 
Seismograms from these events are processed prior to the simulations and filtered to be compatible to
the same parameters of the simulations (f_max). \par


\begin{figure} [!h]
\centering
\includegraphics[scale=1]{fig5.jpg}
\caption{$Epicenter location of the 160 earth- quakes used in the 3D tomographic inversion done by Lee et al. (2013).$ \cite{r5}}
\label{Fig5}
\end{figure}

Step 2. Simulation of events: Simulations are done for the events selected in Step 1 and have a maximum frequency 1 Hz. at this resolution, the differences
between CVM-S and CVM-SI.26 will have a greater
impact on the ground motion?which we seek to quantify in
terms of prediction improvements. \par

Step 3. Post-processing of results and  validation:
Results from the simulations obtained in step 2
will be compared with the data collected and processed in
Step 1. \par

Validation Criteria: \par
For the validation of the different simulations we will use the GOF criterion proposed by Anderson (2004).
This criterion uses ten individual parameters (Ci): Arias duration (C1), energy duration (C2), Arias intensity
(C3), energy integral (C4), peak acceleration (C5), peak velocity (C6), peak displacement (C7), response
spectrum (C8), Fourier amplitude spectrum (C9), and cross correlation (C10). Each parameter is scaled
such that it yields a score varying from 0 to 10, where a score of 10 corresponds to a perfect match between
the two signals for the given metric. Following our previous validation work (Taborda and Bielak, 2013b),
we add an eleventh metric (C11) to explicitly account for the duration of the strong motion phase of the
earthquake and combine them all using the rule: \par



 \begin{equation}\label{z1}
 \begin{split}
 & S = (\frac{1}{9})\times((\frac{1}{2})(C _{1}+ C_{2}) +( \frac{1}{2})(C_{3} + C_{4}) )
 \end{split}
 \end{equation}


Although there are other GOF and misfit criteria available in the literature (e.g. Kristekova et al., 2006,
2009), we preferred the method introduced by Anderson (2004) because its metrics convey physical meaning
to both seismologists and engineers. During the comparative validation process we may still choose other
typical metrics such as biases of individual measurements (i.e., peak ground response) or the arrival time of
P-waves. These additional metrics are useful because they provide signed values for the comparisons.

\section{Results: }
Here, We have the result of quantitative measures to assess the level
of agreement between the data and the simulation, using different parameters, scores, and combinations
proposed by Anderson (2004) with the modifications
explained earlier. We begin by presenting the overall final
result, and then break it down into some of its different components
and combinations\par
Figure ... shows the distribution
of S1 scores, averaged between the three components of motion
for all the validation stations in the region of interest for any of the events.\par


\begin{figure} [!h]
\centering
\includegraphics[scale=1]{fig6.jpg}
\caption{$Total score in different frequency ranges.$ \cite{r6}}
\label{Figure 6}
\end{figure}

According to GOF criteria, when comparing two
given signals, a score under 4 should be considered as a poor
fit, a score from 4 to 6 as a fair fit, a score from 6 to 8 as a
good fit, and a score above 8 as an excellent fit. Table 2 represents the average of total score of all the stations of each events. Following
this convention, we characterize the overall agreement between
the simulation and the records from the earthquake
as fair, with the majority (55.7\%) of the stations in the upper
end of this rank, yielding S1 values above 5.5, and 31.8\% of
the stations scoring above 6, in the good-fit category.\par
%  (the numbers need to be fixed according to our work. )

\begin{figure} [!h]
\centering
\includegraphics[scale=1]{fig7.jpg}
\caption{$Total score graph for CVMS & CVMS426 & difference.$ \cite{r7}}
\label{Figure 7}
\end{figure}

In general, the spatial distribution of the scores does not
suggest any direct relationship with the geology of the
region, nor does it follow a definite pattern.\par

\section{Discussion and conclusions: }

The main research goal of the proposed activities is to evaluate the velocity model CVM-SI.26, which integrates the
perturbations obtained from a 3D tomographic inversion based on the original model CVM-S.





\section{Acknowledgements: }

This work was supported by the ? and the Southern California Earthquake Center through ?


\section{Using the GJI class file}

If the file \verb"gji.cls" is not already in the appropriate
system directory for \LaTeX\ files, either arrange for it to be
put there, or copy it to your working directory. The class file
and related material, such as this guide, can be accessed via the
journal web-site  at
http://www.blackwellpublishing.com/journals/gji under {\em Author
Guidelines}.

The GJI document class is implemented as a complete document class, {\em
not\/} a document class option. In   order to use the GJI style, replace
\verb"article" by
\verb"gji" in the
\verb"\documentclass" command at the beginning of your document:
\begin{verbatim}
\documentclass{article}
\end{verbatim}
is replaced by
\begin{verbatim}
\documentclass{gji}
\end{verbatim}
In general, the following standard document class options should {\em
not\/} be used with the GJI style:
\begin{enumerate}
  \item \texttt{10pt}, \texttt{11pt}, \texttt{12pt} -- unavailable;
  \item \texttt{twoside} (no associated style file) --
     \texttt{twoside} is the default;
  \item \texttt{fleqn}, \texttt{leqno}, \texttt{titlepage} --
        should not be used (\verb"fleqn" is already incorporated into
        the GJI style);
  \item \texttt{twocolumn} -- is not necessary as it is the default style.
\end{enumerate}

In \LaTeX2e the use of postscript fonts and the inclusion of non-standard
options is carried out through the \verb"\usepackage" command, rather than
as options as in earlier versions.  Thus the Times font can be used for
text by including
\begin{verbatim}
\usepackage{times}
\end{verbatim}
on the line immediately after the \verb"\documentclass". If necessary,
\texttt{ifthen} and \texttt{bezier} can be included as packages.

The GJI class file has been designed to operate with the standard
version of \verb"lfonts.tex" that is distributed as part of \LaTeX
. If you have access to the source file for this guide,
\verb"gjilguid2e.tex", attempt to typeset it.  If you find font
problems you might investigate whether a non-standard version of
\verb"lfonts.tex" has been installed in your system.

\subsection{Additional document class options}\label{classoptions}

The following additional class options are available with the GJI style:
\begin{description}
  \item \texttt{onecolumn} -- to be used \textit{only} when two-column output
        is unable to accommodate long equations;
  \item \texttt{landscape} -- for producing wide figures and tables which
        need to be included in landscape format (i.e.\ sideways) rather
        than portrait (i.e.\ upright). This option is described below.
  \item \texttt{doublespacing} -- this will double-space your
        article by setting \verb"\baselinestretch" to 2.
  \item \texttt{referee} -- 12/20pt text size, single column,
        designed for submission of papers.
  \item \texttt{mreferee} -- 11/17pt text size, single column
        designed for submission of papers with mathematical content.
  \item \texttt{camera} -- designed for use with computer modern fonts to
        produce a closer representation of GJI style for camera
        ready material.
  \item \texttt{galley} -- no running heads, no attempt to align
        the bottom of columns.
\end{description}


\subsection{Landscape pages}

If a table or illustration is too wide to fit the standard measure, it
must be turned, with its caption, through 90 degrees anticlockwise.
Landscape illustrations and/or tables cannot be produced directly
using the GJI style file because \TeX\ itself cannot turn the
page, and not all device drivers provide such a facility.
The following procedure can be used to produce such pages.
\begin{enumerate}
  \item Use the \verb"table*" or \verb"figure*" environments in your
        document to create the space for your table or figure on the
        appropriate page of your document. Include an empty
        caption in this environment to ensure the correct
        numbering of subsequent tables and figures. For instance, the
        following code prints a page with the running head, a message
        half way down and the figure number towards the bottom. If you
        are including a plate, the running headline is different, and you
        need to key in the three lines which are marked with \verb"% **",
        with an appropriate headline.
\begin{verbatim}
% ** \clearpage
% ** \thispagestyle{plate}
% ** \plate{Opposite p.~812, GJI, \textbf{135}}
\begin{figure*}
  \vbox to220mm{\vfil Landscape figure to
                go here. \vfil}
  \caption{}
  \label{landfig}
\end{figure*}
\end{verbatim}
\item Create a separate document with the corresponding document style
      but also with the \verb"landscape" document style option, and
      include the \verb"\pagestyle" command, as follows:
\begin{verbatim}
\documentclass[landscape]{gji}
\pagestyle{empty}
\end{verbatim}
  \item Include your complete tables and illustrations (or space for
        these) with captions using the \verb"table*" and \verb"figure*"
        environments.
  \item Before each float environment, use the
        \verb"\setcounter" command to ensure the correct numbering of
        the caption. For example,
\begin{verbatim}
\setcounter{table}{0}
\begin{table*}
 \begin{minipage}{115mm}
 \caption{Images of global seismic tomography.}
 \label{tab1}
 \begin{tabular}{@{}llllcll}
   :
 \end{tabular}
 \end{minipage}
\end{table*}
\end{verbatim}
The corresponding example for a figure would be:
\begin{verbatim}
\clearpage
\setcounter{figure}{12}
\begin{figure*}
 \vspace{144mm}
 \caption{Travel times for regional model.}
 \label{fig13}
\end{figure*}
\end{verbatim}
\end{enumerate}


\section{Additional facilities}

In addition to all the standard \LaTeX\ design elements, the GJI style
includes the following features.
\begin{enumerate}
  \item Extended commands for specifying a short version of the title and
        author(s) for the running headlines;
  \item A \verb"summary" environment to produce a suitably indented
        Summary
  \item An \verb"abstract" environment which produces the GJI style of
        Summary
  \item A \verb"keywords" environment and a \verb"\nokeywords" command;
  \item Use of the \verb"description" environment for unnumbered lists.
  \item A starred version of the \verb"\caption" command to produce
        captions for continued figures or tables.
 \end{enumerate}
 In general, once you have used the additional \verb"gji.cls" facilities
in your document, do not process it with a standard \LaTeX\ style file.

\subsection{Titles and author's name}

In the GJI style, the title of the article and the author's name (or
authors' names) are used both at the beginning of the article for the
main title and throughout the article as running headlines at the top
of every page. The title is used on odd-numbered pages (rectos) and the
author's name appears on even-numbered pages (versos). Although the
main heading can run to several lines of text, the running headline
must be a single line ($\leqslant 45$ characters). Moreover, the main
heading can also incorporate new line commands (e.g. \verb"\\") but
these are not acceptable in a running headline. To enable you to
specify an alternative short title and an alternative short author's
name, the standard \verb"\title" and \verb"\author" commands have been
extended to take an optional argument to be used as the running
headline. The running headlines for this guide were produced using the
following code:
\begin{verbatim}
\title[Geophys.\ J.\ Int.:
       \LaTeXe\ Guide for Authors]
  {Geophysical Journal International:
   \LaTeXe\ style guide for authors}
\end{verbatim}
and
\begin{verbatim}
\author[B.L.N. Kennett]
   {B.L.N. Kennett$^1$
  \thanks{Pacific Region Office, GJI} \\
  $^{1}$Research School of Earth Sciences,
    Australian National University,
    Canberra ACT \emph{0200}, Australia
  }
\end{verbatim}
The \verb"\thanks" note produces a footnote to the title or author.

\subsection{Key words and Summary}

At the beginning of your article, the title should be generated in the
usual way using the \verb"\maketitle" command. Immediately following
the title you should include a Summary followed by a list of key
words. The summary should be enclosed within an \verb"summary"
environment, followed immediately by the key words enclosed in a
\verb"keywords" environment. For example, the titles for this guide
were produced by the following source:
\begin{verbatim}
\maketitle
\begin{summary}
 This guide is for authors who are preparing
 papers for \textit{Geophysical Journal
 International} using the \LaTeXe\ document
 preparation system and the GJI style file.
\end{summary}
\begin{keywords}
 \LaTeXe\ -- class files: \verb"gji.cls"\ --
 sample text -- user guide.
\end{keywords}

\section{Introduction}
  :
\end{verbatim}
The heading `\textbf{Key words}' is included automatically and the key
words are followed by vertical space. If, for any reason, there are no
key words, you should insert the \verb"\nokeywords" command immediately
after the end of the \verb"summary" or \verb"abstract" environment. This
ensures that the   vertical space after the abstract and/or title is
correct and that any
\verb"thanks" acknowledgments are correctly included at the bottom of
the first column. For example,
\begin{verbatim}
\maketitle
\begin{abstract}
  :
\end{abstract}
\nokeywords

\section{Introduction}
  :
\end{verbatim}

Note that the \verb"summary" and \verb"abstract" environments have the same
effect for the documentclass \verb"gji.cls"

\subsection{Lists}

The GJI style provides numbered lists using the \verb"enumerate"
environment and unnumbered lists using the \verb"description"
environment with an empty label. Bulleted lists are not part of the GJI
style and the \verb"itemize" environment should not be used.

The enumerated list numbers each list item with roman numerals:
\begin{enumerate}
  \item first item
  \item second item
  \item third item
\end{enumerate}
Alternative numbering styles can be achieved by inserting a
redefinition of the number labelling command after the
\verb"\begin{enumerate}". For example, the list
\begin{enumerate}
\renewcommand{\theenumi}{(\arabic{enumi})}
  \item first item
  \item second item
  \item etc\ldots
\end{enumerate}
was produced by:
\begin{verbatim}
\begin{enumerate}
 \renewcommand{\theenumi}{(\arabic{enumi})}
  \item first item
       :
\end{enumerate}
\end{verbatim}
Unnumbered lists are provided using the \verb"description" environment.
For example,
\begin{description}
  \item First unnumbered item which has no label and is indented from
        the left margin.
  \item Second unnumbered item.
  \item Third unnumbered item.
\end{description}
was produced by,
\begin{verbatim}
\begin{description}
 \item First unnumbered item...
 \item Second unnumbered item.
 \item Third unnumbered item.
\end{description}
\end{verbatim}

\subsection{Captions for continued figures and tables}

The \verb"\caption*" command may be used to produce a caption with the
same number as the previous caption (for the corresponding type of
float). For instance, if a very large table does not fit on one page,
it must be split into two floats; the second float should use the
\verb"caption*" command with a suitable caption:
\begin{verbatim}
\begin{table}
 \caption*{-- \textit{continued}}
  \begin{tabular}{@{}lccll}
  :
  \end{tabular}
\end{table}
\end{verbatim}

 \begin{figure}
     \vspace{5.5cm}
     \caption{An example figure in which space has been
              left for the artwork.}
     \label{sample-figure}
  \end{figure}

\section[]{Some guidelines for using\\* standard facilities}

The following notes may help you achieve the best effects with the GJI
style file.

\subsection{Sections}

\LaTeX\ provides five levels of section headings and they are all
defined in the GJI style file:
\begin{description}
  \item \verb"\section"
  \item \verb"\subsection"
  \item \verb"\subsubsection"
  \item \verb"\paragraph"
  \item \verb"\subparagraph"
\end{description}
Section numbers are given for section, subsection, subsubsection
and paragraph headings.  Section headings are automatically converted to
upper case; if you need any other style, see the example in section~\ref{headings}.

If you find your section/subsection (etc.)\ headings are wrapping round,
you must use the \verb"\\*" to end individual lines and include the
optional argument \verb"[]" in the section command. This ensures that
the turnover is flushleft.

\subsection{Illustrations (or figures)}

\begin{figure*}
 \vspace{5.5cm}
     \caption{An example figure spanning two-columns
             in which space has been left for the artwork.}
     \label{twocol-figure}
\end{figure*}

The GJI style will cope with positioning of your illustrations and
you should not use the positional qualifiers on the
\verb"figure" environment which would override these decisions. See
`Instructions for Authors' in {\em Geophysical Journal International\/}
for submission of
artwork. Figure captions should be below the figure itself, therefore
the \verb"\caption" command should appear after the figure or space
left for an illustration. For example, Fig.~\ref{sample-figure} is
produced using the following commands:
\begin{verbatim}
\begin{figure}
 \vspace{5.5cm}
 \caption{An example figure in which space has
          been left for the artwork.}
 \label{sample-figure}
\end{figure}
\end{verbatim}

Where a figure needs to span two-columns the \verb"figure*" environment
should be used as in  Fig.~\ref{twocol-figure} using the following commands
\begin{verbatim}
\begin{figure*}
 \vspace{5.5cm}
   \caption{An example figure spanning two-columns
     in which space has been left for the artwork.}
   \label{twocol-figure}
\end{figure*}
\end{verbatim}

\subsection{Tables}

The GJI style will cope with positioning of your tables and you
should not use the positional qualifiers on the
\verb"table" environment which would override these decisions. Table
captions should be at the top, therefore the \verb"\caption" command
should appear before the body of the table.

The \verb"tabular" environment can be used to produce tables with
single horizontal rules, which are allowed, if desired, at the head and
foot only. This environment has been modified for the GJI style in the
following ways:
\begin{enumerate}
  \item additional vertical space is inserted on either side of a rule;
  \item vertical lines are not produced.
\end{enumerate}
Commands to redefine quantities such as \verb"\arraystretch" should be
omitted. For example, Table~\ref{symbols} is produced using the
following commands.
\begin{table}
 \caption{Seismic velocities at major discontinuities.}
 \label{symbols}
 \begin{tabular}{@{}lcccccc}
  Class & depth & radius
        & $\alpha _{-}$ & $\alpha _{+}$
        & $\beta _{-}$ & $\beta _{+}$ \\
  ICB & 5154 & 1217 & 11.091 & 10.258
        & 3.438 &  0. \\
  CMB & 2889 & 3482 & 8.009 & 13.691
        & 0. & 7.301 \\
 \end{tabular}

 \medskip
 The ICB represents the boundary between the inner and outer cores and
the CMB the boundary between the core and the mantle.  Velocities with
subscript $-$ are evaluated just below the discontinuity and
those with subscript $+$ are evaluated just above the discontinuity.
\end{table}
\begin{verbatim}
\begin{table}
 \caption{Seismic velocities at major
          discontinuities.}
 \label{symbols}
 \begin{tabular}{@{}lcccccc}
  Class & depth & radius
        & $\alpha _{-}$ & $\alpha _{+}$
        & $\beta _{-}$ & $\beta _{+}$ \\
  ICB & 5154 & 1217 & 11.091 & 10.258
        & 3.438 &  0. \\
  CMB & 2889 & 3482 & 8.009 & 13.691
        & 0. & 7.301 \\
 \end{tabular}

 \medskip
 The ICB represents the boundary ...
... evaluated just above the discontinuity.

\end{table}
\end{verbatim}

If you have a table that is to extend over two columns, you need to use
\verb"table*" in a minipage environment, i.e., you can say
\begin{verbatim}
\begin{table*}
\begin{minipage}{80mm}
 \caption{Caption which will wrap round to the
          width of the minipage environment.}
 \begin{tabular}{%
      :
 \end{tabular}
\end{minipage}
\end{table*}
\end{verbatim}
The width of the minipage should more or less be the width of your table,
so you can only guess on a value on the first pass. The value will have to
be adjusted when your article is finally typeset, so don't worry
about making it the exact size.

\subsection{Running headlines}

As described above, the title of the article and the author's name (or
authors' names) are used as running headlines at the top of every page.
The headline on right pages can list up to three names; for more than
three use et~al. The \verb"\pagestyle" and \verb"\thispagestyle"
commands should {\em not\/} be used. Similarly, the commands
\verb"\markright" and \verb"\markboth" should not be necessary.

\subsection{Typesetting mathematics}

\subsubsection{Displayed mathematics}

The GJI style will set displayed mathematics flush with the left margin,
provided that you use the \LaTeX\ standard of open and closed square brackets
as delimiters. The equation
\[
 \sum_{i=1}^p \lambda_i =
{\mathrm{trace}}(\mathbf{S})
\]
was typeset in the GJI style using the commands
\begin{verbatim}
\[
 \sum_{i=1}^p \lambda_i =
{\mathrm{trace}}(\mathbf{S})
\]
\end{verbatim}
This correct positioning should be compared with that for
the following centred equation,
$$ \alpha_{j+1} > \bar{\alpha}+ks_{\alpha} $$
which was (wrongly) typeset using double dollars as follows:
\begin{verbatim}
$$ \alpha_{j+1} > \bar{\alpha}+ks_{\alpha} $$
\end{verbatim}
Note that \verb"\mathrm" will produce a roman character
in math mode.

For numbered equations use the \verb"equation" and \verb"eqnarray"
environments which will give the correct positioning.
If equation numbering by section is required the command
\verb"\eqsecnum" should appear after \verb"begin{document}"
at the head of the file.

\subsubsection{Bold math italic}\label{boldmathitalic}

The class file provides a font \verb"\mitbf" defined as:
\begin{verbatim}
\newcommand{\mitbf}[1]{
  \hbox{\mathversion{bold}$#1$}}
\end{verbatim}
Which can be used as follows, to typset the equation
\begin{equation}
  d(\mitbf{{s_{t_u}}}) = \langle [RM(\mitbf{{x_y}}
  + \mitbf{{s_t}}) - RM(\mitbf{{x_y}})]^2 \rangle
\end{equation}
the input should be
\begin{verbatim}
\begin{equation}
  d(\mitbf{{s_{t_u}}}) = \langle [RM(\mitbf{{x_y}}
  + \mitbf{{s_t}}) - RM(\mitbf{{x_y}})]^2 \rangle
\end{equation}
\end{verbatim}

If you are using version 1 of the New Font Selection Scheme, you may
have some messages in your log file that read something like ``Warning:
Font/shape `cmm/b/it' in size~\hbox{$< \!\! 9 \!\! >$} not available
on input line 649. Warning: Using external font `cmmi9' instead on input
line 649.'' If you have such messages, your system will have substituted
math italic characters where you wanted bold math italic ones: you
are advised to upgrade to version 2.


\subsubsection{Bold Greek}\label{boldgreek}

To get bold Greek you use the same
method as for bold math italic. Thus you can input
\begin{verbatim}
\[ \mitbf{{\alpha_{\mu}}} =
\mitbf{\Theta} \alpha. \]
\end{verbatim}
to typeset the equation
\[ \mitbf{{\alpha_{\mu}}} = \mitbf{\Theta} \alpha . \]


\subsection{Points to note in formatting text}\label{formtext}

A number of text characters require special attention
so that \LaTeX\ can properly format a file.

The following characters must be preceded by a
backslash or \LaTeX\ will interpret them as commands:
\begin{quote}
~~~~~~~~~\$~~~\&~~~\%~~~\#~~~\_~~~\{~~~and~~~\}
\end{quote}
must be typed
\begin{center}
\begin{quote}
~~~~~~\verb"\$"~~~\verb"\&"~~~\verb"\%"~~~\verb"\#"
~~~\verb"\_"~~~\verb"\{"~~~and~~~\verb"\}".
\end{quote}
\end{center}

\LaTeX\ interprets all double quotes as closing quotes.
Therefore quotation marks must be typed as pairs of
opening and closing single quotes, for example,
\texttt{ ``quoted text.''}

Note that \LaTeX\ will not recognize greater than or
less than symbols unless they are typed within math
commands (\verb"$>$" or \verb"$<$").

\subsubsection{Special symbols}

The macros for the special symbols in Tables~\ref{mathmode}
and~\ref{anymode}
have been taken from the Springer Verlag `Astronomy and Astrophysics'
design, with their permission. They are directly compatible and use the
same macro names.
These symbols will work in all text sizes, but are only guaranteed to work
in text and displaystyles. Some of the symbols will not get any smaller
when they are used in sub- or superscripts, and will therefore be
displayed at the wrong size. Don't worry about this as the typesetter
will be able to sort this out.
%
\begin{table*}
\begin{minipage}{106mm}
\caption{Special symbols which can only be used in math mode.}
\label{mathmode}
\begin{tabular}{@{}llllll}
Input & Explanation & Output & Input & Explanation & Output\\
\hline
\verb"\la"     & less or approx       & $\la$     &
  \verb"\ga"     & greater or approx    & $\ga$\\[2pt]
\verb"\getsto" & gets over to         & $\getsto$ &
  \verb"\cor"    & corresponds to       & $\cor$\\[2pt]
\verb"\lid"    & less or equal        & $\lid$    &
  \verb"\gid"    & greater or equal     & $\gid$\\[2pt]
\verb"\sol"    & similar over less    & $\sol$    &
  \verb"\sog"    & similar over greater & $\sog$\\[2pt]
\verb"\lse"    & less over simeq      & $\lse$    &
  \verb"\gse"    & greater over simeq   & $\gse$\\[2pt]
\verb"\grole"  & greater over less    & $\grole$  &
  \verb"\leogr"  & less over greater    & $\leogr$\\[2pt]
\verb"\loa"    & less over approx     & $\loa$    &
  \verb"\goa"    & greater over approx  & $\goa$\\
\hline
\end{tabular}
\end{minipage}
\end{table*}
%
\begin{table*}
\begin{minipage}{115mm}
\caption{Special symbols which don't have to be
used in math mode.}
\label{anymode}
\begin{tabular}{@{}llllll}
Input & Explanation & Output & Input & Explanation & Output\\
\hline
\verb"\sun"      & sun symbol            & $\sun$     &
  \verb"\earth"     & earth symbol         & $\earth$   \\[2pt]
\verb"\degr"     & degree                &$\degr$     &
  \verb"\micron"   & \micron               & \micron    \\[2pt]
\verb"\diameter" & diameter              & \diameter  &
  \verb"\sq"       & square                & \squareforqed\\[2pt]
\verb"\fd"       & fraction of day       & \fd        &
  \verb"\fh"       & fraction of hour      & \fh\\[2pt]
\verb"\fm"       & fraction of minute    & \fm        &
  \verb"\fs"       & fraction of second    & \fs\\[2pt]
\verb"\fdg"      & fraction of degree    & \fdg       &
  \verb"\fp"       & fraction of period    & \fp\\[2pt]
\verb"\farcs"    & fraction of arcsecond & \farcs     &
  \verb"\farcm"    & fraction of arcmin    & \farcm\\[2pt]
\verb"\arcsec"   & arcsecond             & \arcsec    &
  \verb"\arcmin"   & arcminute             & \arcmin\\

\hline
\end{tabular}
\end{minipage}
\end{table*}


The command \verb"\chemical" is provided to set chemical species with
an even level for subscripts (not produced in standard mathematics mode).
Thus \verb"\chemical{Fe_{2}^{2+}Cr_{2}O_{4}}" will produce
\chemical{Fe_{2}^{2+}Cr_{2}O_{4}}.


\subsection{Bibliography}

Two methods are provided for managing citations and
references.   The first approach uses the
\verb"\begin{thebibliography}{}"
and \verb"\end{thebibliography}{}" commands.

The second approach uses a simplified scheme using
\verb"\begin{references}" and \verb"\end{references}" commands.

References to published literature should be quoted in text by author
and date; e.g. Draine (1978) or (Begelman, Blandford \& Rees 1984).
Where more than one reference is cited having the same author(s) and date,
the letters a,b,c, \ldots\ should follow the date; e.g.\ Smith (1988a),
Smith (1988b), etc.
The first time you introduce a three-author paper, you should list all
three authors at the first citation, and thereafter, use et al.

\subsubsection{Biblography References in the text}

References in the text are given by author and date, and, whichever
method is used to produce the bibliography, the references in the text
are done in the same way. Each bibliographical entry has a key, which
is assigned by the author and used to refer to that entry in the text.
There is one form of citation -- \verb"\cite{key}" -- to produce the
author and date, and another form -- \verb"\shortcite{key}" -- which
produces the date only. Thus, Rutherford \& Hawker \shortcite{rh} is
produced by
\begin{verbatim}
Rutherford \& Hawker \shortcite{rh}
\end{verbatim}
while \cite{hi} is produced by
\begin{verbatim}
\cite{hi}
\end{verbatim}

\subsubsection{The bibliography}

The following listing shows some references prepared in the style of
the journal; the code produces the references at the end of this guide.
The following rules apply for the ordering of your references:
\begin{enumerate}
  \item if an author has written several papers, some with other authors,
        the rule is that the single-author papers precede the two-author
        papers, which, in turn, precede the multi-author papers;
  \item within the two-author paper citations, the order is determined
        by the second author's surname, regardless of date;
  \item within the multi-author paper citiations, the order is
        chronological, regardless of author's surnames.
\end{enumerate}
%
\begin{verbatim}
\begin{thebibliography}{}
  \bibitem[\protect\citename{Butcher }1992]{bu}
    Butcher J. 1992. \textit{Copy-editing: The
    Cambridge Handbook}, 3rd edn, Cambridge
    Univ. Press, Cambridge.
  \bibitem[\protect\citename{The Chicago Manual }%
    1982]{cm} \textit{The Chicago Manual of Style},
    Univ. Chicago Press, Chicago, 1982.
  \bibitem[\protect\citename{Chao }1985]{ch}
    Chao, B. F., 1985. Normal mode study of the
    Earth's rigid body motions,
    \textit{Geophys. Res. Lett.}, \textbf{12}, 526-529.
  \bibitem[\protect\citename{Hinderer }1986]{hi}
    Hinderer, J., 1986. Resonance effects of the
    earth's fluid core in earth rotation,
    in \textit{Solved and Unsolved Problems},
    pp. 277-296, ed. Cazenave A., Reidel,
    Dordrecht.
  \bibitem[\protect\citename{Lamport }1986]{la}
    Lamport L., 1986,  \LaTeX: \textit{A Document
    Preparation System}, Addison--Wesley, New York
  \bibitem[\protect\citename{Lindberg }1986]{li}
    Lindberg, C., 1986.  Multiple taper harmonic
    analysis of terrestrial free oscillations,
    \textit{PhD thesis}, University of California.
  \bibitem[\protect\citename{Maupin }1992]{ma}
    Maupin, V., 1992. Modelling of laterally
    trapped surface waves with application to
    Rayleigh waves in the Hawaiian swell,
    \textit{Geophys. J. Int.}, \textbf{110}, 553-570.
  \bibitem[\protect\citename{Rutherford
    \& Hawker }1981]{rh} Rutherford, S. R.
    \& Hawker, K. E.,  1981, Consistent coupled
    mode theory of sound propagation for a
    class of non-separable problems,
    \textit{J. acoust. Soc. Am.}, \textbf{71},
    554-564
\end{thebibliography}
\end{verbatim}
Each entry takes the form
\begin{verbatim}
\bibitem[\protect\citename{Author(s), }%
  Date]{tag} Bibliography entry
\end{verbatim}
where \verb"Author(s)" should be the author names as they are cited in
the text, \verb"Date" is the date to be cited in the text, and
\verb"tag" is the tag that is to be used as an argument for the
\verb"\cite{}" and \verb"\shortcite{}" commands. \verb"Bibliography entry"
should be the material that is to appear in the bibliography,
suitably formatted.

\subsubsection{Simplified References and Citations}

The second approach to referencing is taken with permission from
the American Geophysical Union Latex macros

The reference section is started using a
\verb"\begin{references}" command which will
automatically produce a correctly formatted
``Reference'' head.  Each reference is then
preceded by a \verb"\reference"
command.  It is the author's
responsibility to place bibliographic reference
information in the proper order with correct
punctuation.  After the last reference in your
reference section, type an \verb"\end{references}"
command.

Authors may enter properly formatted citations directly
in the manuscript text and enclose those citations in
\verb"\markcite{}" commands.  This approach
marks all citations in your manuscript, but there
is no interaction between the \verb"\markcite"
commands and the reference section.

To create in-text citations, enclose each citation
within a \verb"\markcite" command.
There are two ways to include in-text citations,
depending on the way you phrase your sentence.
You may either include an entire reference within
brackets \markcite{(Merritt et al., 1996)} or you
may mention the author as part of your sentence and
include only the year in brackets, as in \markcite{Ono (1996)}.

As an example
\begin{verbatim}
\begin{references}
\reference
Azimi, Sh.A., Kalinin, A.Y., Kalinin, V.B.,
\& Pivovarov, B.L., 1968.
Impulse and transient characteristics of media
with linear and quadratic absorption laws,
\textit{Izv. Earth Phys.} (English Transl.),
\textbf{2}, 88--93.
\reference
Dahlen, F.A., \& Smith, M.L., 1975.
The influence of rotation on the free
oscillations of the Earth,
\textit{Phil. Trans. R. Soc. London Ser. A},
\textbf{279}, 143--167.
\reference
Durek, J.J., Ritzwoller, M.H.,
\& Woodhouse, J.H., 1993.
Constraining upper mantle anelasticity
using surface wave amplitude anomalies,
\gji, \textbf{114}, 249--272.
\end{references}
\end{verbatim}
produces the reference list
\begin{references}
 \reference
Azimi, Sh.A., Kalinin, A.Y., Kalinin, V.B.,
\& Pivovarov, B.L., 1968.
Impulse and transient characteristics of media
with linear and  quadratic absorption laws,
\textit{Izv. Earth Phys.} (English Transl.),
\textbf{2}, 88--93.
\reference
Dahlen, F.A., \& Smith, M.L., 1975.
The influence of rotation on the free oscillations of the Earth,
\textit{Phil. Trans. R. Soc. London Ser. A}, \textbf{279}, 143--167.
\reference
Durek, J.J., Ritzwoller, M.H., \& Woodhouse, J.H., 1993.
Constraining upper mantle anelasticity
using surface wave amplitude anomalies,
\gji \textbf{114}, 249--272.
\end{references}

\subsubsection{Common Journals}

The following abbreviations are provided for commonly cited
journals and can be used directly in the bibliography.

In the following table the abbreviation and the form of the
associated entry are presented
\newline
\begin{tabular}{ll}
\verb"\areps" & \areps \\
\verb"\bssa"  & \bssa \\
\verb"\eos"   & \eos  \\
\verb"\eps"   & \eps \\
\verb"\epsl"  & \epsl \\
\verb"\gca"   & \gca \\
\verb"\geo"   & \geo \\
\verb"\geop"  & \geop \\
\verb"\gji"   & \gji \\
\verb"\gjras" & \gjras \\
\verb"\grl"   & \grl \\
\verb"\gsab"  & \gsab \\
\verb"\gs"    & \gs \\
\verb"\jgr"   & \jgr \\
\verb"\jseis" & \jseis \\
\verb"\mnras" & \mnras \\
\verb"\pag"   & \pag \\
\verb"\pepi"  & \pepi \\
\verb"\rg"    & \rg \\
\verb"\tecto" & \tecto \\
\end{tabular}
%
% The following two tables have been moved back in the text to
% improve page layout
%
\begin{table*}
\begin{minipage}{130mm}
\caption{Authors' notes.}
\label{authors}
\begin{tabular}{@{}ll}
\verb"\title[optional short title]{long title}"
                    & short title used in running head\\
\verb"\author[optional short author(s)]{long author(s)}"
                    & short author(s) used in running head\\
\verb"\begin{abstract}...\end{abstract}"& for summary on
titlepage\\
\verb"\begin{summary}...\end{summary}"& for abstract on
titlepage\\
\verb"\begin{keywords}...\end{keywords}"& for keywords on titlepage\\
\verb"\nokeywords"  & if there are no keywords on titlepage\\
\verb"\begin{figure*}...\end{figure*}" & for a double spanning figure in two-column mode\\
\verb"\begin{table*}...\end{table*}" & for a double spanning table in
                                       two-column mode\\
\verb"\caption*"    & for continuation figure captions\\
\verb"\resetfigno" & resets figures numbers after an appendix\\
\verb"[referee]" & documentclass option for 12/20pt, single col,
                   for manuscript submission\\
\verb"[mreferee]" & documentclass option for 11/17pt, single col,
                   for submission of papers with extensive mathematics\\
\end{tabular}
\end{minipage}
\end{table*}
%
\begin{table*}
\begin{minipage}{130mm}
\caption{Editors' notes.}
\label{editors}
\begin{tabular}{@{}lp{270pt}}
\verb"\pagerange{000--000}"& for catchline, note use of en-rule\\
\verb"\pagerange{L00--L00}"& for letters option, used in catchline\\
\verb"\volume{000}" & volume number, for catchline\\
\verb"\pubyear{0000}" & publication year, for catchline\\
\verb"\microfiche{GJI000/0}" & for articles accompanied by microfiche\\
\verb"\journal" & replace the whole catchline at one go\\
\verb"[doublespacing]" & documentclass option for doublespacing\\
\verb"[galley]" & documentclass option for running to galley\\
\verb"[landscape]" & documentclass option for landscape illustrations\\
\verb"[fasttrack]" & documentclass option, for rapid short communications
                   (adds F to folios)\\
\verb"[onecolumn]" & documentclass option for one-column \\
\verb"\bsp"& typesets the final phrase `This paper has been produced
 using the Blackwell Publishing GJI \LaTeX2e\ class file.'\\
\end{tabular}
\end{minipage}
\end{table*}

\subsection{Appendices}

The appendices in this guide were generated by typing:
\begin{verbatim}
\appendix
\section{For authors}
     :
\section{For editors}
\end{verbatim}
You only need to type \verb"\appendix" once. Thereafter, every
\verb"\section" command will generate a new appendix which will be
numbered A, B, etc.

If figure captions are to provided after an appendix the figure number can
be reset to avoid extraneous labelling using the command
\verb"\resetfigno".

\section[]{Example of section heading with\\*
  {\mdseries \textsc{S}\lowercase{\textsc{mall}}
  \textsc{C}\lowercase{\textsc{aps}}},
  \lowercase{lowercase},
  \textit{ italic}, and bold\\* Greek such as
  $\mitbf{{\mu^\kappa}}$}\label{headings}

This can be built up using text commands and the \verb"mitbf"
command introduced above

\begin{verbatim}
\section[]{Example of section heading with\\*
  {\mdseries \textsc{S}\lowercase{\textsc{mall}}
  \textsc{C}\lowercase{\textsc{aps}}},
  \lowercase{lowercase},
  \textit{ italic}, and bold\\* Greek such as
  $\mitbf{{\mu^\kappa}}$}\label{headings}
\end{verbatim}

\subsection{Acknowledgments}
Acknowledgments after the main text and before the appendices can be
included with the
\texttt{acknowledgments} environment, as
\begin{verbatim}
\begin{acknowledgments}
We wish to thank ...
\end{acknowledgments}
\end{verbatim}
There is also a corresponding \texttt{acknowledgment} environment for a
single acknowledgment.

\begin{acknowledgments}
A number of colleagues have helped with suggestions for the
improvement of this material and I would particularly like to thank
Bob Geller, University of Tokyo for his criticisms and corrections.
\end{acknowledgments}

\begin{thebibliography}{}
  \bibitem[\protect\citename{Frankel }1992]{bu}
    Frankel A. \& Vidale J. 1992. \textit{A three-dimensional simulation of
seismic waves in the Santa Clara Valley, California, from a
Loma Prieta aftershock}, Seismol. Soc. Am. 82, no. 5, 2045?
2074.
    
  \bibitem[\protect\citename{Graves }1998]{bu}
    Graves R. W.  \textit{Three-dimensional finite-difference modeling of
the San Andreas fault: Source parameterization and ground-motion
levels}, Bull. Seismol. Soc. Am. 88, no. 4, 881?897.
 
  \bibitem[\protect\citename{Olsen }1995]{bu}
    Olsen K. B. \& Archuleta J.  \& Matarese R.1995. \textit{Threedimensional
simulation of a magnitude 7.75 earthquake on the
San Andreas fault}, Science 270, no. 5242, 1628?1632.
  
  \bibitem[\protect\citename{Bao }1998]{bu}
    Bao H. \& Bielak J.  \& Ghattas O. \& Kallivokas F. \& O'Hallaron D. R. \& Shewchuk J. R. \& Xu J. 1998. \textit{Large-scale simulation of elastic
wave propagation in heterogeneous media on parallel computers}, Comput. Meth. Appl. Mech. Eng. 152, no. 1?2, 85?102.
    
  \bibitem[\protect\citename{Seriani }1998]{bu}
    Seriani G. 1998. \textit{3-D large-scale wave propagation modeling by spectral
element method on Cray T3E multiprocessor}, Comput. Meth. Appl. Mech. Eng. 164, no. 1?2, 235?247.
    
 
  \bibitem[\protect\citename{Kaser }2006]{bu}
    Kaser G. \& Cogley J. G. \& Dyurgerov M. B \& Meier M. F. \& Ohmura A. 2006. \textit{Mass balance of glaciers and ice caps: consensus estimates for 1961?2004}, Geophysical Research Letters, 33(19).
 
  
  \bibitem[\protect\citename{Komatitsch }2004]{bu}
    Komatitsch D. \& Liu Q. \& Tromp J. \& Suss P. \& Stidham C. \& Shaw J. H. 2004. \textit{Simulations of ground motion in the Los Angeles basin based upon the spectral-element method}, Bull. Seismol. Soc. Am. 94, no. 1, 187?206.
  
  
  \bibitem[\protect\citename{Aagaard }2008]{bu}
    Aagaard B. T. \& Brocher T. M. \& Dolenc D. \& Dreger D. \& Graves R. W. \& Harmsen S. \& Hartzell S. \& Larsen S. \& Zoback M. L. 2008. \textit{Groundmotion
modeling of the 1906 San Francisco earthquake, Part I: Validation using the 1989 Loma Prieta earthquake}, Bull. Seismol. Soc. Am. 98, no. 2, 989?1011.
 
  \bibitem[\protect\citename{Aagaard }2008]{bu}
    Aagaard B. T. \& Brocher T. M. \& Dolenc D. \& Dreger D. \& Graves R. W. \& Harmsen S. \& Hartzell S. \& Larsen S. \& Zoback M. L. 2008. \textit{Groundmotion
modeling of the 1906 San Francisco earthquake, Part I: Validation using the 1989 Loma Prieta earthquake}, Bull. Seismol. Soc. Am. 98, no. 2, 989?1011.
    
  \bibitem[\protect\citename{Graves }2008]{bu}
    Graves R. W. \& Aagaard B. T. \& Hudnut K. W. \& Star L. M. \& Stewart J. P. \& Jordan T. H.  2008. \textit{Broadband simulations forMw 7.8 southern San Andreas earthquakes: Ground motion sensitivity to rupture speed}, Geophys. Res. Lett. 35, no. 22, L22302, doi: 10.1029/2008GL035750.
    
  \bibitem[\protect\citename{Olsen }2010]{bu}
    Olsen K. B. \& Mayhew J. E.  2010. \textit{Goodness-of-fit criteria for broadband synthetic seismograms, with application
to the 2008 Mw 5.4 Chino Hills}, California, earthquake, Seismol. Res. Lett. 81 (5): 715?723.
    
  \bibitem[\protect\citename{Bielak }2010]{bu}
    Bielak J. \& Graves R. W. \&Olsen K. B. \& Taborda R. \& Ram�rez-Guzm�n L. \& Day L. \& Ely S. M. \& Roten D. \& Jordan T. H. \&  Maechling P. J. \& Urbanic J. \& Cui Y. \& Juve G. 2010. \textit{The ShakeOut Earthquake Scenario: Verification
of Three Simulation Sets}, Geophys. J. Int. 180 (1): 375?404.    
    
  \bibitem[\protect\citename{Cui }2010]{bu}
    Cui Y. \& Olsen K. B \&Olsen K. B. \&  Jordan T. H. \&  Lee K. \& Zhou J. \& Small P. \& Roten D. \& Ely G. \& Panda D. K. \& Chourasia A. \& Levesque J. \& Day S. M. \& Maechling P.  2010. \textit{Scalable earthquake simulation on petascale
supercomputers, in High Performance Computing, Networking, Storage and Analysis (SC)}, 2010 International Conference, New Orleans,
Louisiana, 13?19 November 2010, 1?20.    

   
  \bibitem[\protect\citename{Taborda }2013]{bu}
    Taborda  R. 2013. \textit{}, 2013 ...  
   
      \bibitem[\protect\citename{Lee }2014]{bu}
    Lee  En-Jui \& Chen P. \& Jordan T. H.. 2014. \textit{Testing waveform prediction of 3D velocity models against two recent Los Angeles earthquakes}, Seismological Research Letters, 85(6), 1275-1284.     
    
    
        
  
          
  \bibitem[\protect\citename{The Chicago Manual }%
    1982]{cm} \textit{The Chicago Manual of Style}, Univ.
    Chicago Press, Chicago, 1982.
  \bibitem[\protect\citename{Chao }1985]{ch}
    Chao, B. F., 1985. Normal mode study of the Earth's rigid
     body motions, \textit{Geophys. Res. Lett.}, \textbf{12}, 526-529.
  \bibitem[\protect\citename{Hinderer }1986]{hi}
    Hinderer, J., 1986. Resonance effects of the earth's fluid
    core in earth rotation, in \textit{Solved and Unsolved
    Problems}, pp. 277-296, ed. Cazenave A., Reidel,
    Dordrecht.
  \bibitem[\protect\citename{Kopka \& Daly}1995]{kd}
    Kopka H. \& Daly P.W., 1995, \textit{A guide to} \LaTeX2e,
    Addison--Wesley, New York
  \bibitem[\protect\citename{Lamport }1986]{la}
    Lamport L., 1986,  \LaTeX: \textit{A Document
    Preparation System}, Addison--Wesley, New York
  \bibitem[\protect\citename{Lindberg }1986]{bl}
    Lindberg, C., 1986.  Multiple taper harmonic analysis of
    terrestrial free oscillations, \textit{PhD thesis},
    University of California.
  \bibitem[\protect\citename{Maupin }1992]{ma}
    Maupin, V., 1992. Modelling of laterally trapped surface
    waves with application to Rayleigh waves in the Hawaiian
    swell, \textit{Geophys. J. Int.}. \textbf{110}, 553-570.
  \bibitem[\protect\citename{Rutherford \& Hawker }%
    1981]{rh}  Rutherford, S. R. \& Hawker, K. E., 1981,
    Consistent coupled mode theory of sound propagation for a
    class of non-separable problems,
   \textit{J. acoust. Soc. Am.}, \textbf{71}, 554-564
\end{thebibliography}


\appendix
\section{For authors}

Table~\ref{authors} is a list of design macros which are unique to GJI. The
list displays each macro's name and description.

\section{For editors}

The additional features shown in Table~\ref{editors} may be used for
production purposes.

\bsp % ``This paper has been produced using the Blackwell
     %   Publishing GJI \LaTeXe\ class file.''

\label{lastpage}


\end{document}
