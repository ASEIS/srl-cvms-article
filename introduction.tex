\section{Introduction}

Developing community velocity models (CVMs) that can be broadly applied and continuously updated by a community of users has became of interest in simulation-based analysis. CVM-S and CVM-H models, released by Southern California Earthquake Center (SCEC), are good examples of community models, evolved over time. CVM-S, also known as CVM-S4, was originally developed by \citet{Magistrale_1996_BSSA} and later updated by \citet{Magistrale_2000_BSSA} and \citet{Kohler_2003_BSSA}. Recently, a new version of CVM-S, called CVM-S4.26 was built, based on the original model and the results of a sequence of 3D full-waveform tomographic inversions done by \citet{Chen_2007_BSSA} and \citet{Lee_2014_JGR}. CVM-S is known as an acceptable representations of the crustal structure in southern California at low frequencies ($f \leq 0.2$~Hz). According to \citet{Taborda_2016} CVM-S is getting better results than the alternative model CVM-H for the region understudy, even upto 1 Hz, which is in good agreement with previous works done in the area for particular cases (e.g. \citet{Taborda_2014_BSSA} and \citet{Lee_2014_SRL}). There are currently three alternative CVM-SI.26 models (2.2.1, 2.2.2 \& 2.2.3) available which vary depending on how the perturbations were applied to the original model. There have been some researches with the aim of evaluation of different velocity models for southern California region in lower frequencies or for limited number of events. Such differences, however, have never been studied extensively for multiple events and/or at frequencies beyond the upper limits set by the underlying inversions used to construct the models.\par
Here, we design a systematic procedure, through quantitative comparisons among synthetics results of four versions of CVM-S and recorded data of thirty moderate-magnitude events, to evaluate the overall improvement of accuracy in predicting ground motion within simulation domain. The simulations are performed at 1 Hz frequency and minimum shear wave velocity of 200 m/s using Hercules, a finite element application for solving forward wave propagation problems due to kinematic faulting \citep{Tu_2006_Proc, Taborda_2010_Tech}. Comparisions are ranked quantitatively by means of a goodness-of-fit (GOF) criteria at 0.5 Hz, obtained from a modified version of the criteria introduced by \citet{Anderson_2004_Proc}. In the light of the comparision of the regional distribution of the GOF results for all events and all models, We conclude that CVM-SI26.223 consistently yields better results which confirms the improvements in this latest versions.
