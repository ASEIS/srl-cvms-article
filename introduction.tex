\section{Introduction}

Among the several alternatives to earthquake ground motion simulations, physics-based, deterministic approach have drawn significant attention from seismologists and earthquake engineers \citep[e.g.][]{Tromp_2010_GJI, Aagaard_2010_BSSAii, Graves_2011_PAG, Isbiliroglu_2015_ES} specially due to the increasing capacity and availability of high-performance computer systems and applications \citep[e.g.][]{Cui_2010_Proc} which make it possible now to have different seismic velocity modesl available for various regions such as southern and northern California in the United States \citep{Kohler_2003_BSSA, Brocher_2006_Proc}; the Grenoble and Volvi valleys in Europe \citep{Chaljub_2010_BSSA, Manakou_2010_SDEE}; and Japan \citep{Koketsu_2008_Proc, Fujiwara_2009_Tech}. \par
Being able to develop models that can be applied more broadly, and that can be continuously updated by a community of users has became the point of interest in simulation-based hazard analysis which lead to making models known as community velocity models, or CVMs. Two series of CVM-S (Magistrale et al., 2000; Kohler et al., 2003), and CVM-H (S�ss and Shaw, 2003) released by Southern California Earthquake Center (SCEC) are good examples of community models. They have evolved over time with contributions from a community of researchers who study the earthquake hazards and ground motion characteristics in southern California.\par
CVM-S, also known as CVM-S4, was originally developed by \citet{Magistrale_1996_BSSA} and later updated by \citet{Magistrale_2000_BSSA} and \citet{Kohler_2003_BSSA}. Recently, a new version of CVM-S, called CVM-S4.26 was built based on the original model CVM-S4 and the results of a sequence of 3D full-waveform tomographic inversions done by \citet{Chen_2007_BSSA} and \citet{Lee_2014_JGR}. CVM-S is known as an acceptable representations of the crustal structure in southern California, at low frequencies ($f \leq 0.2$~Hz) and also according to our current parallel research for evaluating that, even till 1 Hz frequency, it is getting better results than the alternative model CVM-H for the region. Those results are in agreement with previous works done in the area for particular cases (e.g. \citet{Taborda_2014_BSSA},  \citet{Lee_2014_SRL}). There are currently three alternative CVM-SI.26 models (2.2.1, 2.2.2 \& 2.2.3) which vary depending on how the perturbations were applied to the original model. The ability to properly simulate seismograms depends on the challenge of finding the most accurate models of 3D crustal structure. Different models can lead to significantly different results, even at very low frequencies \citep[e.g.][]{Lee_2014_SRL, Taborda_2014_BSSA}. There have been some researches with the aim of evaluation of different velocity models for southern California region in lower frequencies or for limited number of events. Such differences, however, have never been studied extensively, in order to compare different available models at frequencies beyond the upper limits set by the underlying inversions used to construct the models. \par
So, here, we designed a procedure to evaluate the overall improvement of accuracy of this recent southern California velocity model, CVM-SI.26, in predicting ground motion of the region, through comparing the results from different versions of CVM-S.
Four different version of seismic velocity model, including CVM-S4, CVM-SI26.2.2.1, CVM-SI26.2.2.2 \& CVM-SI26.2.2.3, and a set of historical events, containing thirty moderate-magnitude earthquakes ($3.5 > M_w > 5.5$) are considered for conducting a systematic evaluation, using validation of multiple simulations, through quantitative comparisons between synthetics and data. Events occurred between 1998 and 2014 and their are spread throughout the region and within a simulation domain with a surface projection area of \adomain{180}{135}{km}. The simulations are performed using a finite element application for solving forward wave propagation problems due to kinematic faulting \citep{Tu_2006_Proc, Taborda_2010_Tech}, with a numerical model built to represent a maximum frequency, \fmaxeq{1} and a minimum shear wave velocity, \vsmineq{200}. \par
The comparisons between data and synthetics are ranked quantitatively by means of a goodness-of-fit (GOF) criteria obtained following a modified version of the criteria introduced by \citet{Anderson_2004_Proc}. We compare the regional distribution of the GOF results for all events and all models, and draw conclusions from the results and how these correlate to the characteristics of the models. We find that, in light of our comparisons, one of the models consistently yields better results and explore the reasons leading to this conclusion which in addition to agreement with previous case studies by \citet{Taborda_2014_BSSA} and \citet{Lee_2014_SRL}, shows that the improvements done to CVM-S based on the tomographic studies done by \citet{Chen_2007_BSSA} and \citet{Lee_2014_SRL} have positive effects at frequencies above the limit considered for the inversions (0.2~Hz). 