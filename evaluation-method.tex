
\section{Evaluation Method}

Using the goodness-of-fit (GOF) method proposed by \citet{Anderson_2004_Proc} and modified by \citet{Taborda_2013_BSSA}, GOF scores results for all velocity models and earthquakes combinations are prepared. The method is basically based on comparions of synthetics records with data, at locations where records are available. Parameters involved in calculating GOF scores (scaled from 0 to 10 showing improvement with increment) include Arias intensity integral (C1), energy integral (C2), Arias intensity value (C3), total energy (C4), peak acceleration (C5), peak velocity (C6), peak displacement (C7), response spectrum (C8), Fourier amplitude spectrum (C9), cross correlation (C10), and strong-phase duration (C11). After calculating the eleven scores for each pair, the combined score can be obtained from: 
%
\begin{equation}
	\mathrm{S} = \frac{1}{9} \left( 
		\frac{\mathrm{C}1+\mathrm{C}2}{2} +
		\frac{\mathrm{C}3+\mathrm{C}4}{2} +
		\sum_{i=5}^{11} \mathrm{C}i
	\right)
	\hspace{0.25em}.
	\label{eq:s}
\end{equation}

The scoring procedure will be applied to different bands, following the guidelines suggested by \citet{Anderson_2004_Proc}, using compatible ``broadband'' sets, and a series of band-pass filtered versions of the signals or sub-bands, SB$_1$, SB$_2$, and SB$_3$. The results of S from equation (\ref{eq:s}) for the broadband (BB) and each of the sub-bands (SB$_i$) are then combined to obtain a final score (FS), define as:
%
\begin{equation}
	\mathrm{FS} = \frac{1}{4} \left( \mathrm{BB} + \sum^3_{i=1} \mathrm{SB}_i \right)
	\hspace{0.25em}.
	\label{eq:fs}
\end{equation}

We use signals band-pass filtered between 0.1 and 0.5 Hz for BB, and between 0.1 and 0.25~Hz, 0.25 and 0.35~Hz, and 0.35 and 0.5~Hz for SB$_1$, SB$_2$, and SB$_3$, respectively. Although we perform the simulations for 1~Hz frequency,the upper limit of 0.5~Hz is selected to better present the differences between various models. The lower limit is goverened by the need for minimizing instrumental and numerical processing issues at the low frequencies (0--0.1~Hz), which is applied duringcan processing on signals. More over, This is helpful to eliminate permanent displacements in the synthetics extracted at locations near the epicenter, especially for shallower earthquakes. 

