% SRL Article

\documentclass[10pt]{article}
\usepackage{fullpage}
\usepackage{amsmath, amssymb, amscd}
\usepackage{graphicx}
\usepackage{fancyhdr}
\usepackage{comment}
\usepackage{color}

 
 \title{:\par 1. A comparative study between CVM-SI.26 and CVMS-400 Velocity Model \par 2.The Comparison of Two Versions of Southern California Velocity Models: CVM-SI.26 and CVMS-400\par 3. Evaluation of latest Southern California Velocity Models, CVM-SI.26, through Simulation and Validation of Multiple Historical Events\par 4. A comparative study between Two Versions of Southern California Velocity Models: CVMS-400 and the latest one, CVM-SI.26, through Simulation and Validation of Multiple Historical Events }
\author{}
\begin{document}
\maketitle

\section{Introduction: }
Among the several alternatives to earthquake simulations, physics-based, deterministic approach to seismic hazard analysis (SHA) 
have drawn significant attention from seismologists and earthquake engineers in the last few years.   \par
These simulations use numerical approaches such as the finite-element (FE), finite-difference (FD), 
and high-order FE or spectral element methods (e.g., Frankel and Vidale, 1992; Olsen et al., 1995; Graves, 1996; Bao et al., 1998; Seriani, 1998; K�ser and Dumbser, 2006). 
With the growth of high-performance computing (HPC) facilities and applications, the implementation of numerical methods for large scale
three-dimensional (3D) earthquake simulations and rupture dynamics has progressed considerably  and researchers are now modeling earthquake-related processes at a level of fidelity that was thought not possible just a few years ago (e.g., Komatitsch et al., 2004; Aagaard et al., 2008; Graves, 2008; Olsen et al., 2009; Bielak et al., 2010; Cui et al., 2010). \par 

This has allowed us to deliver products of relevant application in seismology and engineering, such as families of synthetics for scenario earthquakes (e.g., Jones et al., 2008) and physics-based seismic hazard maps (e.g., Graves et al., 2011). The complexity and multi-scale nature of earthquakes, however, still poses immense challenges like the accuracy of the models which is heavily depend on the source and velocity models used as input. \par

In recent years, significant advances have been made in 3D waveform tomography (e.g., Tape et al.,
2009; Chen et al., 2007a, 2010; Lee et al., 2011; Lee and Chen 2013). In particular, Chen et al. (2007b) and Lee et
al., (2013) have used local small and medium-sized earthquakes and ambient noise Green's functions in full 3D
waveform tomographic inversions for the crustal structure of the Los Angeles and greater Southern California regions.
Thus far, their work has produced a total of 26 iteration datasets with perturbations to the SCEC Community
Velocity Model (CVM-S). \par

Recently obtained results from a full three-dimensional (3D) tomography inversion for the crustal
structure of the Southern California region provide perturbation datasets that are being integrated back
into the original velocity model used as reference (CVM-S) to build a new model called CVM-SI.26. These perturbations
are, however, available only at a coarse resolution (500 m) and the scheme put forward to integrate them
with CVM-S needs to be tested before the new model CVM-SI.26 can be used for querying information at arbitrary
points.  \par

The main goal here is to provide a better understanding of the improvements in the new model
CVM-SI.26 when compared to the original model CVM-S. This article evaluates the overall improvements obtained with the inversion and tests the correctness of the integration schemes when used at higher resolution than that of the inversion. This evaluation is done systematically through the forward simulation and validation of multiple small- and moderate-sized events using forward simulations in order to avoid biases from individual simulations. 


\section{Material Models: CVMS400 and CVMS426 }

The ability to simulate seismograms depends on having accurate 3D material model. Material models, also called seismic velocity or crustal models, are a key element in the simulation process because, while being used to construct the mesh and determine the properties of mesh elements that constitute the numerical representation of the system,
their properties determine in good part the outcome of the simulation. \par

High-resolution seismic velocity models in urban regions have received increased attention over the last decade due to the interest in earthquake modelling. Several material models
have been developed and improved over the years for the regions of southern and northern California (e.g. Graves 1996b; Magistrale
et al. 1996; Hauksson & Haase 1997; Magistrale et al. 2000; Kohler et al. 2003; S?uss & Shaw 2003; Brocher 2005). \par

We use subsets of SCEC?s Community Velocity Model (CVM), version 4.0, and version 4.26 in a discretized form that is stored in a database library called ?etree?
(Tu et al. 2003); this accelerates considerably the mesh generation process (Schlosser et al. 2008). The CVM was built upon the initial model assembled by
Magistrale et al. (1996), and later extended and improved by Magistrale et al. (2000) and Kohler et al. (2003). In Magistrale et al. (2000), the P-wave velocity (Vp) of the major southern California basins (Los Angeles basin,Ventura basin, San GabrielValley, San Fernando Valley, Chino basin, San Bernardino Valley and the Salton Trough) was determined primarily by application of empirical rules based on depths and ages estimated for four geological horizons and calibrated for southern California with data from deep boreholes. The density and Poisson?s ratio (and S-wave velocity, Vs) were estimated from Vp using empirical relationships. Outside and below the basins, properties were based on 3-D seismic tomography
(Hauksson 2000). Kohler et al. (2003) added upper mantle seismic velocities to the model, based on the inversion of teleseismic traveltime
residuals obtained from three temporary passive experiments and stations of the Southern California Seismic Network. \par

%%%%%%%%%%%%%%%%%%%%%%%%%%%%%%%%%%%%%%%%%%%%%%

The latest release by SCEC, CVM version 4.0, has a new San Bernardino Valley basement, based on recent inversions of gravity data confirmed by comparison to a seismic reflection line. The depth to basement of the Salton Trough was extended to the northern Mexican territory and redefined by a combination of seismic refraction surveys, inversion
of gravity observations, surface geology, and boreholes. CVM version 4.0 also has a new V p-density relationship based on density
measurements from oil well samples in the Los Angeles basin and the San Gabriel Valley, geotechnical boreholes throughout southern
California, and 12 oil wells along the Los Angeles Region Seismic Experiment lines. \par

The CVM is distributed as a Fortran code that reads points specified
by latitude, longitude and depth, and writes out Vp, V s, and
density (?) values at those points. Since the Fortran code has to
combine data from many different associated files, its performance
for the construction of large meshes needed for FDM and FEM
models is poor. To overcome this problem and to unify to the maximum
possible extent the data used for the grids and mesh generation
processes of each modelling group, we stored a discretized version
of the CVM (for the region of interest) in an octant tree database
form, called ?etree? (Tu et al. 2003). The etree library is known
for its efficiency in query retrieval time and has been used successfully
in past earthquake simulations (e.g. Akcelik et al. 2003;
Rodgers et al. 2008). The discretization process, described in detail
in Taborda et al. (2007), was done in a way that guarantees that
the discretized version, hereafter called CVM?Etree, is appropriate
for simulations up to a maximum frequency f max = 1 Hz and a
minimum shear wave velocity Vsmin
= 200 ms?1, with at least eight
points per wavelength (p = 8). The resolution of the CVM-Etree
decreases with depth while maintaining the rule eo ? (Vs/ f max)/p,
where eo is the octant edge size. This results in a minimum octant
at the surface with eo = 18.3 m?where there is maximum
variability?and a maximum octant with eo = 9.2 km at 80 km in
depth?where the crustal structure is fairly homogeneous. Fig. 5
shows the southern California basins and other geological features
as represented in the model, and the basins floor and shear wave
velocity structure of the resulting CVM-Etree at various depths. \par

%%%%%%%%%%%%%%%%%%%%%%%%%%%%%%%%%%%%%


CVM-S is built upon the original work by Magistrale et al.
(1996, 2000). It integrates available information about the
major southern California basins (Los Angeles basin,
Ventura basin, San Gabriel valley, San Fernando valley,
Chino basin, San Bernardino valley, and the Salton trough).
Outside and below the basins, CVM-S uses a 3D seismic
tomography model (Hauksson, 2000) and an upper-mantle
model based on teleseismic inversions (Kohler et al., 2003).
Inside the basins, it uses data from shallow and deep boreholes,
oil wells, gravity observations, seismic refraction surveys
surveys,
and other empirical rules based on the depths and ages
estimated for a set of geological horizons.
When queried at a particular longitude, latitude, and
depth, CVM-S provides the values of VP, VS, and ? for the
specific query point. In general, CVM-S operates by depth,
in reference to the free surface, but does not provide information
about the topography. We ignore the free-surface
elevation and model the region in the simulation domain
by flattening the topography (see squashed topography in
Aagaard et al., 2008). We access the information stored in
CVM-S through the Unified Community Velocity Model
(UCVM) software framework developed by SCEC (Small
et al., 2011; Gill et al., 2013) and use additional utilities provided
by UCVM to facilitate the simulation process (see Simulation
Approach and Models section).\par


Crustal models that specify the lateral and depth variations of seismic velocities have been developed for southern California by synthesizing constraints from geologic studies, well logs, seismic reflection and refraction surveys, and earthquake tomography (Hauksson, 2000; Magistrale et al., 2000; Kohler et al., 2003; S�ss and
Shaw, 2003) \par

\cite{r3} compares sections through two of the SCEC models?CVM-S4 and CVM-S4.26 with 2D tomographic models
derived from a dense collection of reflection/refraction data along the two lines of the Los Angeles region seismic experiments
(LARSE), LARSE I (Lutter et al., 1999) and LARSE II (Lutter et al., 2004). The models show significant differences at
the basin scales, as well as deeper in the crust.

\begin{figure} [!h]
\centering
\includegraphics[scale=1]{fig1.jpg}
\caption{$Comparisons of P-wave velocity models along the LARSE-I (left) and LARSE-II (right) profiles. From top to bottom, we show
the 2D tomography models obtained using controlled-source travel-time data and the cross-section views through CVM-S4.26 and CVM-S4. The vertical axes on the velocity model plots have been exaggerated by a factor of 2.0 for the LARSE-I profile and a factor
of 2.5 for the LARSE-II profile. The geographic maps show the locations of the LARSE-I (left) and LARSE-II (right) profiles and the epicenters
of the Encino (yellow star) and the La Habra (red star) earthquakes. WF, Whittier fault; SMFZ, Sierra Madre fault zone; SAF, San
Andreas fault; SSF, Santa Susana fault; SGF, San Gabriel fault; GF, Garlock fault.$ \cite{r1}}
\label{Figure 1}
\end{figure}


In CVM-S, the seismic velocities within major basins were determined
mainly from the age and depths of the sediments using
empirical relations. The latest official release is version 4
(CVM-S4), which includes a geotechnical layer constrained by
sonic log data (Magistrale et al., 2000), a variable-depth Moho
determined from receiver functions (Zhu and Kanamori, 2000),
and an upper-mantle velocity model from Moho to about
100 km depth (Kohler et al., 2003). \par

CVM-S4.26 is the 26th iterate of a full 3D tomographic
(F3DT) inversion procedure (Lee et al., 2014). The procedure
started with CVM-S4 and successively improved the fit to datasets
that eventually included about 550,000 differential waveform
measurements at frequencies up to 0.2 Hz, obtained from
about 38,000 earthquake seismograms and 12,000 ambientnoise
Green?s functions. Navigation through this nonlinear
iterative process involved two types of F3DT inversion methods:
the AW-F3DT, which backpropagates the misfits between
observed and synthetic seismograms from the receivers to
image structures (Tarantola, 1984; 1988; Pratt, 1990; Tromp
et al., 2005), and the scattering-integral method (SI-F3DT),
which calculates and stores the sensitivity kernels of each misfit
measurement and solves the Gauss?Newton normal equation
using the least-squares algorithm (Zhao et al., 2005, Zhao et al.,
2006; Chen et al., 2007).
In each inversion step, synthetic seismograms for the updated
model were calculated using the Olsen (1994) fourthorder
staggered-grid finite-difference code, which has been
optimized for massively parallel computations (Cui et al.,
2010), and differential waveform measurements were made between
the observed seismograms and these synthetics, accounting
for the nonlinearity of the inversion. High structural
resolution was obtained by included frequency-dependent,
phase-coherent measurements of various seismic phases on all
three components of ground motion. Lee et al. (2014) describe
CVM-S4.26 and demonstrate its excellent fit to observed seismograms
from a large number of well-recorded earthquakes.
The seismograms from the two Los Angeles events analyzed
here were not used in the F3DT inversion. \par




There are currently three alternative CVM-SI.26 models which vary depending on how the perturbations were applied
to the original model. The different versions and the overall process are described in greater detail in the GTL
entry of the SCEC Wiki Web site (http://scec.usc.edu/scecpedia/GTL). The entire process goes beyond the scope of
the proposed research because the ultimate goal is, not only to incorporate the inversion results, but also to extend
the implementation to include geotechnical layers (GTL) used in
other community models (e.g., CVM-H). I refer here only to the
models that reconcile the inversion perturbations?as a first step
in that process, and use the same nomenclature employed in the
Web site. These models are: \par

(2.2.1) This model applies an integration scheme in which
negative perturbations are only used outside the basins, and
positive perturbations are used everywhere.\par

(2.2.2) This scheme applies negative perturbations only if
outside the basins, and disregards positive perturbations
when inside the basins.\par

(2.2.3) This scheme applies both negative and positive perturbations
everywhere, but sets the original value inside the
basins as a floor limit.\par

Here, the term basin refers to any structure with Vs \le1000
m/s adopting the cutoff value used by Lee et al. (2013) to build
the starting model. In all cases, the schemes reverse the process
used to build the starting model first, in order to recapture the
original structures in CVM-S with values of Vs \le 1000 m/s, and
apply the perturbations later (as described). These schemes have
been implemented in the SCEC Unified Community Velocity
Model (UCVM) software framework (Gill et al. 2013), which has
already been used to produce unstructured (etree) meshes (Taborda
et al., 2007; Tu et al., 2003) at resolutions finer than that used in the inversion. Figure 1 shows the Vs profile
at a depth of 50 m in an area that includes all the major basins in the Los Angeles region and its corresponding perturbations1.\par

\begin{figure} [!h]
\centering
\includegraphics[scale=1]{fig2.jpg}
\caption{$Shear wave velocity (in m/s) o the CVM-SI26(2.2.3) model at a depth of 50 m (top) and perturbations with respect to the original CVM-S model (bottom).$ \cite{r2}}
\label{Figure 2}
\end{figure}



\section{Methodology and Parameters:  }

In order to evaluate CVM-SI.26, we need to evaluate the integration scheme and to quantify the improvement (or lack thereof) to the prediction
of the ground motion that results from using the new integrated model. The applied method is to do this
by performing a series of simulations using CVM-SI.26 in contrast to CVM-S for a collection of events. The 29 selected events (3.5 \le M \le 5.5) are a subset of the 160 earthquakes used in the inversion process. The synthetic ground-motion results
from these simulations are compared with records available for these events through the Southern California
Earthquake Data Center (www.data.scec.org). The comparisons are done quantitatively using intensity goodness-of-fit measures  of a modified version of the Anderson Goodness-of-Fit (GOF) method and by direct waveform comparisons.This method combines various comparison criteria that are meaningful to both engineers and seismologists, and has been satisfactorily used in previous validation efforts.  Simulations are done at a maximum resolution frequency higher
than that of the inversion?in order to assess the arbitrariness of the new model?but are kept low enough (0.1 \le
fmax \le 1 Hz) so that it facilitates the execution of as many simulations as possible. \par

\cite{r3}. summarizes the simulation parameters. The etree models were built on Kraken at the National Institute
for Computational Sciences, and the ground-motion simulations were done on Blue Waters at the National Center for
Supercomputing Applications. \cite{r4}.shows the mesh characteristics and simulation performance for each model and simulation.


\begin{figure} [!h]
\centering
\includegraphics[scale=1]{table3.jpg}
\caption{$Selected Events Chractristics$ \cite{r6}}
\label{Table3}
\end{figure}


\begin{figure} [!h]
\centering
\includegraphics[scale=1]{table1.jpg}
\caption{$Simulation Parameters$ \cite{r3}}
\label{Table1}
\end{figure}


\begin{figure} [!h]
\centering
\includegraphics[scale=1]{table2.jpg}
\caption{$Finite-Element Mesh Characteristics and Simulation Performance$ \cite{r4}}
\label{Table2}
\end{figure}


The process includes gathering of data, setup of simulation models, and execution of an
initial simulations which is done at the High Performance Computing Center ofBlue Water which has a cluster large enough to conduct initial simulations.  The approach will consists of simulations of
multiple small- and medium-size events and quantitative validation of synthetics to assess model improvements such as those used in Taborda
and Bielak (2013, 2014).
This requires the use of parallel computing.  The following steps are taken for the process:\par

Step1. Selection of events and data pre-processing: A
total of 160 events were used in the 3D tomographic inversion
done by Lee et al. (2013) (see Figure 4.a). We select a region
similar to that , that is suitable for
evaluating the effects, and select a subset of these events
(~30). Table 1 presents a summary of properties of each of the selected earthquake. 
Seismograms from these events are processed prior to the simulations and filtered to be compatible to
the same parameters of the simulations (f_max). \par


\begin{figure} [!h]
\centering
\includegraphics[scale=1]{fig5.jpg}
\caption{$Epicenter location of the 160 earth- quakes used in the 3D tomographic inversion done by Lee et al. (2013).$ \cite{r5}}
\label{Fig5}
\end{figure}

Step 2. Simulation of events: Simulations are done for the events selected in Step 1 and have a maximum frequency 1 Hz. at this resolution, the differences
between CVM-S and CVM-SI.26 will have a greater
impact on the ground motion?which we seek to quantify in
terms of prediction improvements. \par

Step 3. Post-processing of results and  validation:
Results from the simulations obtained in step 2
will be compared with the data collected and processed in
Step 1. \par

Validation Criteria: \par
For the validation of the different simulations we will use the GOF criterion proposed by Anderson (2004).
This criterion uses ten individual parameters (Ci): Arias duration (C1), energy duration (C2), Arias intensity
(C3), energy integral (C4), peak acceleration (C5), peak velocity (C6), peak displacement (C7), response
spectrum (C8), Fourier amplitude spectrum (C9), and cross correlation (C10). Each parameter is scaled
such that it yields a score varying from 0 to 10, where a score of 10 corresponds to a perfect match between
the two signals for the given metric. Following our previous validation work (Taborda and Bielak, 2013b),
we add an eleventh metric (C11) to explicitly account for the duration of the strong motion phase of the
earthquake and combine them all using the rule: \par



 \begin{equation}\label{z1}
 \begin{split}
 & S = (\frac{1}{9})\times((\frac{1}{2})(C _{1}+ C_{2}) +( \frac{1}{2})(C_{3} + C_{4}) )
 \end{split}
 \end{equation}


Although there are other GOF and misfit criteria available in the literature (e.g. Kristekova et al., 2006,
2009), we preferred the method introduced by Anderson (2004) because its metrics convey physical meaning
to both seismologists and engineers. During the comparative validation process we may still choose other
typical metrics such as biases of individual measurements (i.e., peak ground response) or the arrival time of
P-waves. These additional metrics are useful because they provide signed values for the comparisons.

\section{Results: }
Here, We have the result of quantitative measures to assess the level
of agreement between the data and the simulation, using different parameters, scores, and combinations
proposed by Anderson (2004) with the modifications
explained earlier. We begin by presenting the overall final
result, and then break it down into some of its different components
and combinations\par
Figure ... shows the distribution
of S1 scores, averaged between the three components of motion
for all the validation stations in the region of interest for any of the events.\par


\begin{figure} [!h]
\centering
\includegraphics[scale=1]{fig6.jpg}
\caption{$Total score in different frequency ranges.$ \cite{r6}}
\label{Figure 6}
\end{figure}

According to GOF criteria, when comparing two
given signals, a score under 4 should be considered as a poor
fit, a score from 4 to 6 as a fair fit, a score from 6 to 8 as a
good fit, and a score above 8 as an excellent fit. Table 2 represents the average of total score of all the stations of each events. Following
this convention, we characterize the overall agreement between
the simulation and the records from the earthquake
as fair, with the majority (55.7\%) of the stations in the upper
end of this rank, yielding S1 values above 5.5, and 31.8\% of
the stations scoring above 6, in the good-fit category.\par
%  (the numbers need to be fixed according to our work. )

\begin{figure} [!h]
\centering
\includegraphics[scale=1]{fig7.jpg}
\caption{$Total score graph for CVMS & CVMS426 & difference.$ \cite{r7}}
\label{Figure 7}
\end{figure}

In general, the spatial distribution of the scores does not
suggest any direct relationship with the geology of the
region, nor does it follow a definite pattern.\par

\section{Discussion and conclusions: }

The main research goal of the proposed activities is to evaluate the velocity model CVM-SI.26, which integrates the
perturbations obtained from a 3D tomographic inversion based on the original model CVM-S.





\section{Acknowledgements: }

This work was supported by the ? and the Southern California Earthquake Center through ?

\section{References:}



\end{document}
