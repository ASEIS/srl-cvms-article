% SRL Article

\documentclass[10pt]{article}
\usepackage{fullpage}
\usepackage{amsmath, amssymb, amscd}
\usepackage{graphicx}
\usepackage{fancyhdr}
\usepackage{comment}
\usepackage{color}

 
 \title{:\par 1. A comparative study between CVM-SI.26 and CVMS-400 Velocity Model \par 2.The Comparison of Two Versions of Southern California Velocity Models: CVM-SI.26 and CVMS-400\par 3. Evaluation of latest Sothern California Velocity Models, CVM-SI.26, through Simulation and Validation of Multiple Historical Events\par 4. A comparative study between Two Versions of Sothern California Velocity Models: CVMS-400 and the latest one, CVM-SI.26, through Simulation and Validation of Multiple Historical Events }
\author{}
\date{2015-May-13}
\begin{document}
\maketitle
\newpage


\section{Introduction: }
Earthquake simulations are needed to understand the
propagation of seismic waves, the characteristics of the
ground motion, and as a complement or a substitute to recorded
data. Among the several alternatives to earthquake
simulations, physics-based, deterministic earthquake simulations
have drawn significant attention from seismologists and
earthquake engineers in the last few years.   \par

% how much history for simulation?
(These simulations
use numerical approaches such as the finite-element (FE), finite-
difference (FD), and high-order FE or spectral element
methods (e.g., Frankel and Vidale, 1992; Olsen et al.,
1995; Graves, 1996; Bao et al., 1998; Seriani, 1998; K�ser
and Dumbser, 2006). In particular, thanks to the accelerated
growth of high-performance computing facilities and applications,
the implementation of numerical methods for largescale
three-dimensional (3D) earthquake simulations and
rupture dynamics has progressed considerably in the last
decade (e.g., Komatitsch et al., 2004; Aagaard et al.,
2008; Graves, 2008; Olsen et al., 2009; Bielak et al., 2010) \par 

Most large-scale deterministic earthquake simulations
have been limited to modeling the propagation of lowfrequency
waves, with resolution parameters fixed at maximum
frequencies, fmax ? 1 Hz, and relatively high values
of minimum shear-wave velocity, VSmin
? 500 m=s (e.g.,
Komatitsch et al., 2004; Olsen et al., 2009; Graves and
Aagaard, 2011). This has been due, in part, to the particular
limitations of simulation algorithms and computer codes, and
the available capacity of high-performance computing facilities.
But it has also been due to the common understanding
within the scientific community that the available mathematical
models of the soil layers and crustal structure of the Earth,
and the theory of anelastic wave propagation, are insufficient
to accurately represent the level of scattering and interferences
present at higher frequencies and softer, shallower soil
deposits. Consequently, broadband simulations required for
realistic seismic hazard analysis, engineering applications,
and structural design are increasingly being produced using
hybrid methods. These methods combine stochastic and
deterministic approaches to produce synthetic seismograms
(e.g., Liu et al., 2006; Graves and Pitarka, 2010). They take
into consideration the physics of the earthquake at low
frequencies from the deterministic simulation, and the random
characteristics observed in transient data recorded during
past earthquakes from the stochastic method. \par 
Validation studies of deterministic simulations thus far
have been limited to very low to low maximum frequencies
(fmax  0:1?1:0 Hz), and, in general, they have only used
qualitative, subjective waveform comparisons between
simulation synthetics and recorded seismograms or have narrowly
used quantitative metrics such as biases and residuals
at a reduced number of selected locations (e.g., Hartzell et al.,
2006; Aagaard et al., 2008; Fukuyama et al., 2009). Other
than in the case of simulations that use hybrid approaches,
there have been only a few efforts to explore the validity of
deterministic earthquake ground-motion simulations at
maximum frequencies higher than 1 Hz. Simulations that
consider the effects of soft-soil deposits with shear-wave
velocities as low as 200?300 m=s have also been rare (e.g.,
Taborda et al., 2009; Chaljub et al., 2010). Therefore, there is
a need for exploring up to what frequencies deterministic
simulations continue to be valid. \par )

%%%%%%%%%%%%%%%%%%%%%%%%%%%%%%%%%%

Much progress has been made in earthquake simulation over the last two decades towards a physics-based
approach to seismic hazard analysis (SHA). Thanks to the aid of high-performance computing (HPC) applications,
researchers are now modeling earthquake-related processes at a level of fidelity that was thought
not possible just a few years ago (e.g., Bielak et al., 2010; Cui et al., 2010). This has allowed us to deliver
products of relevant application in seismology and engineering, such as families of synthetics for scenario
earthquakes (e.g., Jones et al., 2008) and physics-based seismic hazard maps (e.g., Graves et al., 2011). The
complexity and multi-scale nature of earthquakes, however, still poses immense challenges. 

One of these
challenges is the accuracy of the models that are used as input in HPC applications for earthquake systems
simulation. 
Current efforts in physics-based SHA, for instance, heavily depend on the source and velocity models
used as input. In particular, solving forward anelastic wave-propagation problems in earthquake simulations
builds upon the accuracy of both these models. 
 \par

In recent years, significant advances have been made in 3D waveform tomography (e.g., Tape et al.,
2009; Chen et al., 2007a, 2010; Lee et al., 2011; Lee and Chen 2013). In particular, Chen et al. (2007b) and Lee et
al., (2013) have used local small and medium-sized earthquakes and ambient noise Green?s functions in full 3D
waveform tomographic inversions for the crustal structure of the Los Angeles and greater Southern California regions.
Thus far, their work has produced a total of 26 iteration datasets with perturbations to the SCEC Community
Velocity Model (CVM-S). \par

Recently obtained results from a full three-dimensional (3D) tomography inversion for the crustal
structure of the Southern California region provide perturbation datasets that are being integrated back
into the original velocity model used as reference (CVM-S) to build a new model called CVM-SI.26. These perturbations
are, however, available only at a coarse resolution (500 m) and the scheme put forward to integrate them
with CVM-S needs to be tested before the new model CVM-SI.26 can be used for querying information at arbitrary
points.  \par

The main goal is to provide a better understanding of the improvements in the new model
CVM-SI.26 when compared to the original model CVM-S. This article evaluates the overall improvements obtained with the inversion and tests the correctness of the integration schemes when used at higher resolution than that of the inversion. This evaluation is done systematically through the forward simulation and validation of multiple small- and moderate-sized events using forward simulations in order to avoid biases from individual simulations. 

\section{Methodology and Parameters:  }
Recent advances in modeling algorithms and methods, and the increasing capability of parallel applications have
made it affordable to solve seismic forward and inverse wave propagation problems over large regions.  \par

As stated in the Science Collaboration Plan, there is an increasing interest in being able to use the results from 
inversions in forward wave propagation simulations at higher frequencies (0.5-5 Hz). To accomplish this, a task group
(including the PI) proposed and implemented various schemes to reconcile the original model (CVM-S) with
the perturbations that resulted from the tomographic inversions (I.26). The resulting models is called CVM-SI.26.
There are currently three alternative CVM-SI.26 models which vary depending on how the perturbations were applied
to the original model. The different versions and the overall process are described in greater detail in the GTL
entry of the SCEC Wiki Web site (http://scec.usc.edu/scecpedia/GTL). The entire process goes beyond the scope of
the proposed research because the ultimate goal is, not only to incorporate the inversion results, but also to extend
the implementation to include geotechnical layers (GTL) used in
other community models (e.g., CVM-H). I refer here only to the
models that reconcile the inversion perturbations?as a first step
in that process, and use the same nomenclature employed in the
Web site. These models are: \par

(2.2.1) This model applies an integration scheme in which
negative perturbations are only used outside the basins, and
positive perturbations are used everywhere.\par

(2.2.2) This scheme applies negative perturbations only if
outside the basins, and disregards positive perturbations
when inside the basins.\par

(2.2.3) This scheme applies both negative and positive perturbations
everywhere, but sets the original value inside the
basins as a floor limit.\par

Here, the term basin refers to any structure with Vs \le1000
m/s?adopting the cutoff value used by Lee et al. (2013) to build
the starting model. In all cases, the schemes reverse the process
used to build the starting model first, in order to recapture the
original structures in CVM-S with values of Vs \le 1000 m/s, and
apply the perturbations later (as described). These schemes have
been implemented in the SCEC Unified Community Velocity
Model (UCVM) software framework (Gill et al. 2013), which has
already been used to produce unstructured (etree) meshes (Taborda
et al., 2007; Tu et al., 2003) at resolutions finer than that used in the inversion. Figure 1 shows the Vs profile
at a depth of 50 m in an area that includes all the major basins in the Los Angeles region and its corresponding perturbations1.\par

In order to evaluate CVM-SI.26,
we need to evaluate the integration scheme and to quantify the improvement (or lack thereof) to the prediction
of the ground motion that results from using the new integrated model. The applied method is to do this
by performing a series of simulations using CVM-SI.26 in contrast to CVM-S for a collection of events. The 29 selected events (3.5 \le M \le5.5) are a subset of the 160 earthquakes used in the inversion process. The synthetic ground-motion results
from these simulations are compared with records available for these events through the Southern California
Earthquake Data Center (www.data.scec.org). The comparisons are done quantitatively using intensity goodness-of-fit measures  of a modified version of the Anderson Goodness-of-Fit (GOF) method and by direct waveform comparisons.This method combines various comparison criteria that are meaningful to both engineers and seismologists, and has been satisfactorily used in previous validation efforts.  Simulations are done at a maximum resolution frequency higher
than that of the inversion?in order to assess the arbitrariness of the new model?but are kept low enough (0.1 \le
fmax \le 1 Hz) so that it facilitates the execution of as many simulations as possible. \par

The process includes gathering of data, setup of simulation models, and execution of an
initial simulations?which is done at the High Performance Computing Center ofBlue Water which has a cluster large enough to conduct initial simulations. The The approach will consists of simulations of
multiple small- and medium-size events and quantitative validation of synthetics to assess model improvements such as those used in Taborda
and Bielak (2013, 2014).
This activities will require the use of parallel computing. The PI
has extensive experience working with parallel applications for
earthquake ground motion simulation and is one of the main developers
of Hercules. \par

Step1. Selection of events and data pre-processing: A
total of 160 events were used in the 3D tomographic inversion
done by Lee et al. (2013) (see Figure 4.a). We select a region
similar to that , that is suitable for
evaluating the effects, and select a subset of these events
(~30). Seismograms from these events are processed prior to the simulations and filtered to be compatible to
the same parameters of the simulations (fmax). \par

Step 2. Simulation of events: Simulations are done for the events selected in Step 1 and have a maximum frequency 1 Hz. at this resolution, the differences
between CVM-S and CVM-SI.26 will have a greater
impact on the ground motion?which we seek to quantify in
terms of prediction improvements. \par

Step 3. Post-processing of results and  validation:
Results from the simulations obtained in step 2
will be compared with the data collected and processed in
Step 1. \par

Validation Criteria: \par
For the validation of the different simulations we will use the GOF criterion proposed by Anderson (2004).
This criterion uses ten individual parameters (Ci): Arias duration (C1), energy duration (C2), Arias intensity
(C3), energy integral (C4), peak acceleration (C5), peak velocity (C6), peak displacement (C7), response
spectrum (C8), Fourier amplitude spectrum (C9), and cross correlation (C10). Each parameter is scaled
such that it yields a score varying from 0 to 10, where a score of 10 corresponds to a perfect match between
the two signals for the given metric. Following our previous validation work (Taborda and Bielak, 2013b),
we add an eleventh metric (C11) to explicitly account for the duration of the strong motion phase of the
earthquake and combine them all using the rule: \par

S = (\frac{1}{9})\times((\frac{1}{2})(C _{1}+ C_{2}) +( \frac{1}{2})(C_{3} + C_{4}) )

Although there are other GOF and misfit criteria available in the literature (e.g. Kristekov?a et al., 2006,
2009), we preferred the method introduced by Anderson (2004) because its metrics convey physical meaning
to both seismologists and engineers. During the comparative validation process we may still choose other
typical metrics such as biases of individual measurements (i.e., peak ground response) or the arrival time of
P-waves. These additional metrics are useful because they provide signed values for the comparisons.

\section{Results: }

\section{Discussion and conclusions: }

The main research goal of the proposed activities is to evaluate the velocity model CVM-SI.26, which integrates the
perturbations obtained from a 3D tomographic inversion based on the original model CVM-S.

\section{Acknowledgements: }

This work was supported by the ? and the Southern California Earthquake Center through ?

\section{References:}



\end{document}
